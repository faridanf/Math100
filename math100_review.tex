\documentclass{beamer}
\mode<presentation>{}
%\usepackage{beamerthemesplit} 

\setbeamertemplate{footline}[frame number]
%\setbeamertemplate{headline}{}


% PACKAGES
%\usepackage{amsthm}
% \usepackage{amsmath}
% \usepackage[margin=1in]{geometry}
% \usepackage{amsfonts}
% \usepackage{graphicx}
 \usepackage{mdframed}
%
%% THEOREMS AND STRUCTURES
%\newtheorem{theorem}{Theorem}[section]
%\newtheorem{corollary}[theorem]{Corollary}
%\newtheorem{lemma}[theorem]{Lemma}
%\newtheorem{Example}[theorem]{Example}
%\newtheorem{proposition}[theorem]{Proposition}
%\numberwithin{equation}{section}
%\newtheorem*{definition}{Definition}
%\newtheorem*{solution}{{\bf Solution}}
%\newtheorem*{remark}{Remark}

\usepackage{dirtytalk}
\usepackage{blindtext}
\usepackage{tcolorbox}
\usepackage{graphicx}
\definecolor{moonstoneblue}{rgb}{0.45, 0.66, 0.76}



\usepackage[utf8]{inputenc}


%Information to be included in the title page:
\title{MATH 100}
\author{Farid Aliniaeifard}
\institute{\bf University of British Columbia}
\date{2019}



\begin{document}
	
	\frame{\titlepage}
	
	
	\begin{frame}
		\frametitle{$
	f(x)=\begin{cases}
	2x & x<3\\
	9 & x=3\\
	2x & x>3
	\end{cases}
	$}
		$$
	\includegraphics[width=1\linewidth]{lim1}
	$$
\end{frame}
\begin{frame}
\frametitle{$f(x)={\rm sin}(\frac{\pi}{x})$}
	$$
\includegraphics[width=1\linewidth]{lim4}
$$
\end{frame}

\begin{frame}
\frametitle{$
	f(x)=	\begin{cases}
		x & x<2\\
		-1 & x=2\\
		x+3 & x>2
		\end{cases}
		$
	}
		$$
		\includegraphics[width=1\linewidth]{lim8}
		$$
\end{frame}
\begin{frame}
\begin{columns}
\begin{column}{0.48\textwidth}
\begin{Example} Consider the graph of the function $f(x)$.
	$$
	\includegraphics[width=.5\linewidth]{lim9}
	$$
	Then 
	$$
	\underset{x\rightarrow 1^-}{\rm lim}f(x)= $$
$$	\underset{x\rightarrow 1^+}{\rm lim}f(x)=  $$ 	$$\underset{x\rightarrow 1}{\rm lim}f(x)=
	$$
\end{Example}
\end{column}


	\begin{column}{0.48\textwidth}
\begin{Example} Consider the graph of the function $g(t)$.
	$$
	\includegraphics[width=.5\linewidth]{lim10}
	$$
	Then 
	$$
	\underset{t\rightarrow 1^-}{\rm lim}g(t)=$$ 
$$	\underset{t\rightarrow 1^+}{\rm lim}g(t)= $$ 
$$	\underset{t\rightarrow 1}{\rm lim}g(t)=
	$$
\end{Example}
\end{column}
\end{columns}
\end{frame}

\begin{frame}
\begin{columns}
	\begin{column}{0.48\textwidth}
		\begin{Example} Consider the graph of the function $f(x)$.
			$$
			\includegraphics[width=.5\linewidth]{lim9}
			$$
			Then 
			$$
			\underset{x\rightarrow 1^-}{\rm lim}f(x)=2 $$
			$$	\underset{x\rightarrow 1^+}{\rm lim}f(x)= 2 $$ 	$$\underset{x\rightarrow 1}{\rm lim}f(x)=2
			$$
		\end{Example}
	\end{column}
	
	
	\begin{column}{0.48\textwidth}
		\begin{Example} Consider the graph of the function $g(t)$.
			$$
			\includegraphics[width=.5\linewidth]{lim10}
			$$
			Then 
			$$
			\underset{t\rightarrow 1^-}{\rm lim}g(t)=2$$ 
			$$	\underset{t\rightarrow 1^+}{\rm lim}g(t)= -2$$ 
			$$	\underset{t\rightarrow 1}{\rm lim}g(t)=DNE
			$$
		\end{Example}
	\end{column}
\end{columns}
\end{frame}

\begin{frame}{When the limit goes to infinity}
\begin{columns}
	\begin{column}{.48\textwidth}
	\begin{Example}
		Consider the graph for the function $f(x)$.
		$$
		\includegraphics[width=.6\linewidth]{lim11}
		$$
		$$
		%	\underset{x\rightarrow a^-}{\rm lim}f(x)=+\infty  \quad \quad 
		%	\underset{x\rightarrow a^+}{\rm lim}f(x)= +\infty \quad \quad
		\underset{x\rightarrow a}{\rm lim}f(x)=+\infty
		$$
	\end{Example}
	\end{column}
	\begin{column}{.48\textwidth}
	
	\begin{Example}
		Consider the graph for the function $g(x)$.
		$$
		\includegraphics[width=.6\linewidth]{lim12}
		$$
		$$
		%	\underset{x\rightarrow a^-}{\rm lim}g(x)=-\infty  \quad \quad 
		%	\underset{x\rightarrow a^+}{\rm lim}g(x)= -\infty \quad \quad  
		\underset{x\rightarrow a}{\rm lim}g(x)=-\infty
		$$
	\end{Example}
\end{column}
\end{columns}
\end{frame}


\begin{frame}
\begin{columns}
	\begin{column}{0.48\textwidth}
		\begin{Example}
			Consider the graph for the function $h(x)$.
			$$
			\includegraphics[width=.6\linewidth]{lim13}
			$$
			$$
			\underset{x\rightarrow a^-}{\rm lim}h(x)= $$$$
			\underset{x\rightarrow a^+}{\rm lim}h(x)=  $$
		\end{Example}	
	\end{column}
	\begin{column}{0.48\textwidth}
		\begin{Example}
			Consider the graph for the function $s(x)$.
			$$
			\includegraphics[width=.6\linewidth]{lim14}
			$$
			$$
			\underset{x\rightarrow a^-}{\rm lim}s(x)= $$ $$
			\underset{x\rightarrow a^+}{\rm lim}s(x)= $$  
		\end{Example}
	\end{column}
\end{columns}


\end{frame}

\begin{frame}
\begin{columns}
	\begin{column}{0.48\textwidth}
			\begin{Example}
			Consider the graph for the function $h(x)$.
			$$
			\includegraphics[width=.6\linewidth]{lim13}
			$$
			$$
			\underset{x\rightarrow a^-}{\rm lim}h(x)=+\infty $$$$
			\underset{x\rightarrow a^+}{\rm lim}h(x)= 3 $$
		\end{Example}	
	\end{column}
	\begin{column}{0.48\textwidth}
		\begin{Example}
			Consider the graph for the function $s(x)$.
			$$
			\includegraphics[width=.6\linewidth]{lim14}
			$$
			$$
			\underset{x\rightarrow a^-}{\rm lim}s(x)=3 $$ $$
			\underset{x\rightarrow a^+}{\rm lim}s(x)=-\infty $$  
		\end{Example}
\end{column}
\end{columns}
	

\end{frame}


\begin{frame}
\begin{Example}
	Consider the function $$g(x)=\frac{1}{sin(x)}.$$ Find the one-side limits of this function as $x\rightarrow \pi$.
\end{Example}
\begin{columns}
	\begin{column}{.48\textwidth}
	$$
	\includegraphics[width=1\linewidth]{sin}
	$$
	\end{column}
\begin{column}{.48\textwidth}
		$$\underset{x\rightarrow \pi^-}{\rm lim}\frac{1}{sin(x)}=+\infty $$$$ \underset{x\rightarrow \pi^+}{\rm lim}\frac{1}{sin(x)}=-\infty \quad $$ 
\end{column}
\end{columns}


\end{frame}

%================NEW SESSION===============
\begin{frame}{\bf Second Session Outline}
\begin{itemize}
	\item Arithmetic of the Limits
	\item {{ Limit of a ratio: what will happen if the limit of the denominator is zero. For example, 
	}}
	 $$\lim_{x \to 0} \frac{1}{x^2}? \quad \text{and} \quad \lim_{x\to 1} \frac{x^3-x^2}{x-1}=? $$
	\item  Sandwich/ Squeeze/Pinch Theorem
	\item limit at infinity
\end{itemize}
\end{frame}

\begin{frame}
\begin{center}
	{\bf \Large \color{blue}Arithmetic of the Limits}
	\end{center}
\end{frame}


\begin{frame}
\begin{tcolorbox}[width=\textwidth,colback={blue!10},title={},colbacktitle=yellow,coltitle=blue] 
	\begin{theorem}
		Let $a,c\in \mathbb{R}$. The following two limits hold 
		$$  	\underset{x\rightarrow a}{\rm lim}c=c \quad \quad 	\underset{x\rightarrow a}{\rm lim}x=a $$
	\end{theorem}
\end{tcolorbox}

	\begin{tcolorbox}[width=\textwidth,colback={green!10},title={},colbacktitle=yellow,coltitle=blue] 
		\begin{example}
	$$\underset{x\rightarrow 3}{\rm lim}-2=-2\quad \quad \quad \underset{x\rightarrow -1}{\rm lim}x=-1$$
\end{example}
\end{tcolorbox}
\end{frame}






\begin{frame}
\begin{tcolorbox}[width=\textwidth,colback={blue!10},title={},colbacktitle=yellow,coltitle=blue] 
	\begin{theorem}({\bf Arithmetic of Limits})
		Let $a,c\in \mathbb{R}$, let $f(x)$ and $g(x)$ be defined for all $x$'s that lie in some interval about $a$ (but $f$ and $g$ need not to be defined exactly at $a$).
		$$\underset{x\rightarrow a}{\rm lim}f(x)=F \quad \quad \underset{x\rightarrow a}{\rm lim}g(x)=G$$ 
		exists with $F,G\in \mathbb{R}$. Then the following limits hold
		\begin{itemize}
			\item 	$ \underset{x\rightarrow a}{\rm lim}(f(x)+g(x))=F+G$--limit of the sum is the sum of the limits. \pause
			\item $ \underset{x\rightarrow a}{\rm lim}(f(x)-g(x))=F-G$--limit of the difference is the difference of the limits. \pause 
			\item $\underset{x\rightarrow a}{\rm lim}cf(x)=cF$. \pause 
			\item $ \underset{x\rightarrow a}{\rm lim}(f(x).g(x))=F.G$--limit of the product is the product of the limits. 
		\end{itemize}
	\end{theorem}
\end{tcolorbox}
\end{frame}

\begin{frame}
\begin{tcolorbox}[width=\textwidth,colback={blue!10},title={},colbacktitle=yellow,coltitle=blue] 

			 If $G\neq 0$ then $$\underset{x\rightarrow a}{\rm lim}\frac{f(x)}{g(x)}=\frac{F}{G}$$

\end{tcolorbox}
\end{frame}
\begin{frame}
\begin{tcolorbox}[width=\textwidth,colback={green!10},title={},colbacktitle=yellow,coltitle=blue] 
\begin{Example}
	Given 
	$$\underset{x\rightarrow 1}{\rm lim}f(x)=3\quad \text{and} \quad \underset{x\rightarrow 1}{\rm lim}g(x)=2 $$
	We have 
	$$\underset{x\rightarrow 1}{\rm lim}3f(x)=\pause 3\times \lim_{x\to 1}f(x)=3\times 3=9.$$
	$$\underset{x\rightarrow 1}{\rm lim}3f(x)-g(x)=\pause 3\times \lim_{x\to 1}f(x)-\lim_{x\to 1}g(x)=3\times 3- 2=7.$$
	$$\underset{x\rightarrow 1}{\rm lim}f(x)g(x)=\pause \lim_{x\to 1}f(x).\lim_{x\to 1}g(x)=3\times 2=6.$$
	$$\underset{x\rightarrow 1}{\rm lim} \frac{f(x)}{f(x)-g(x)}=\pause \frac{\lim_{x\to 1}f(x)}{\lim_{x\to 1}f(x)-\lim_{x\to 1}g(x)}=\frac{3}{3-2}=3.$$
\end{Example}
\end{tcolorbox}
\end{frame}
\begin{frame}{Example}
%\begin{Example} 

$$\lim_{x\to 3}4x^2-1=$$
$$\lim_{x\to 2}\frac{x}{x-1}=$$

%\end{Example}
\end{frame}
\begin{frame}{Example}
%\begin{Example} 

$$\lim_{x\to 3}4x^2-1=4\times \lim_{x\to 3}x^2-\lim_{x\to 3}1=35.$$
$$\lim_{x\to 2}\frac{x}{x-1}=\frac{\lim_{x\to 2}x}{\lim_{x\to 2}x-\lim_{x\to 1}1}=\frac{2}{2-1}=2.$$

%\end{Example}
\end{frame}

\begin{frame}
\begin{center}
	{\bf \Large \color{blue} Limit of a ratio: what will happen if the limit of the denominator is zero.}
\end{center}
\end{frame}

\begin{frame}{{\bf Limit of a ratio: what will happen if the limit of denominator is zero:}}
\begin{itemize}
	\item[--] the limit does {\bf not exist}, eg. 
	$$  \lim_{x\to 0} \frac{x}{x^2}=\lim_{x\to 0} \frac{1}{x}=DNE$$ \pause
	\item[--] the {\bf limit is $\pm \infty$}, eg. 
	$$\lim_{x\to 0} \frac{x^2}{x^4}=\lim_{x\to 0} \frac{1}{x^2}=+\infty \quad \quad \text{or} \quad \quad \lim_{x\to 0} \frac{-x^2}{x^4}=\lim_{x\to 0} \frac{-1}{x^2}=-\infty.$$\pause 
	\item[--] the {\bf limit is $0$}, eg. $$\lim_{x\to 0} \frac{x^2}{x}=\lim_{x\to 0} x=0.$$ \pause 
	\item[--] the {\bf limit exists and it nonzero}, eg. $$\lim_{x\to 0} \frac{x}{x}=1.$$
\end{itemize}
\end{frame}

\begin{frame}

\begin{tcolorbox}[width=\textwidth,colback={blue!10},title={},colbacktitle=yellow,coltitle=blue]  
	\begin{theorem}
		Let $n$ be a positive integer, let $a\in R$ and let $f$ be a function so that
		$$\lim_{x\to a}f(x)=F$$ for some real number $F$. Then the following holds
		$$ \lim_{x\to a}(f(x))^n=\left( \lim_{x\to a} f(x) \right)^n=F^n$$ so that the limit of a power is the power of the limit. \pause Similarly, if 
		\begin{itemize}
			\item $n$ is an even number and $F>0$, or 
			\item $n$ is an odd number and $F$ is any real number 
		\end{itemize}
		then 
		$$ \lim_{x\to a}(f(x))^{1/n}=\left( \lim_{x\to a}f(x) \right)^{1/n}=F^{1/n}.$$
	\end{theorem}
\end{tcolorbox}

\end{frame}

\begin{frame}{Example}
	$$\lim_{x\to 4}x^{1/2}=$$
	$$\lim_{x\to 4}(-x)^{1/2}=$$
	$$ \lim_{x\to 2} (4x^2-3)^{1/3}=$$
\end{frame}

\begin{frame}{Example}
$$\lim_{x\to 4}x^{1/2}=4^{1/2}=2.$$
$$\lim_{x\to 4}(-x)^{1/2}=-4^{1/2}=\text{not a real number}.$$
$$ \lim_{x\to 2} (4x^2-3)^{1/3}=(4(2)^2-3)^{1/3}=(13)^{1/3}.$$
\end{frame}

\begin{frame}
\begin{center}
	 		{\bf \Large  \color{blue} Limit of a ratio: what will happen if the limit of the numerator and denominator are zero,
	 		for example, $$ \lim_{x\to 1} \frac{x^3-x^2}{x-1}=?$$}
 		\end{center}
\end{frame}

\begin{frame}
$$\lim_{x \to 1} \frac{x^3-x^2}{x-1}=?$$
	$$
\includegraphics[width=1\linewidth]{lim18}
$$
\end{frame}

\begin{frame}
\begin{tcolorbox}[width=\textwidth,colback={blue!10},title={},colbacktitle=yellow,coltitle=blue] 
\begin{theorem}
	If $f(x)=g(x)$ except when $x=a$ then $$\lim_{x\to a} f(x)=\lim_{x\to a} g(x)$$ provided the limit of $g$ exists. 
\end{theorem}
\end{tcolorbox}
		$$ \frac{x^3-x^2}{x-1}=\begin{cases}
	x^2 & x\neq 1\\
	\text{undefined} & x=1.
	\end{cases} \Rightarrow  \lim_{x\to 1} \frac{x^3-x^2}{x-1}=\lim_{x\to 1} {x^2}=1. 
	$$
	$$
\includegraphics[width=1\linewidth]{lim18}
$$
\end{frame}

\begin{frame}
\begin{center}
	{\bf \Large \color{blue} Sandwich/ Squeeze/Pinch Theorem}
\end{center}
\end{frame}

\begin{frame}

\begin{Example}
	Compute 
	$$\lim_{x\to 0} x^2\sin(\frac{\pi}{x})$$
\end{Example}
$$
\includegraphics[width=1\linewidth]{lim20}
$$
\end{frame}


\begin{frame}

\begin{Example}
	Let $f(x)$ be a function such that $1\leq f(x)\leq x^2-2x+2$. What is 
	$$\lim_{x\to 1}f(x)?$$
\end{Example}
\pause 
\begin{solution}
	Consider that 
	$$\lim_{x\to 1}x=1\quad \quad \quad \text{and}\quad \quad \quad \lim_{x\to 1}x^2-2x+2=1.$$
	Therefore, by the sandwich/pinch/squeeze theorem 
	$$\lim_{x\to 1}f(x)=1.$$
\end{solution}
\end{frame}

\begin{frame}
\begin{Example}
	We want to compute 
	$$\lim_{x\to +\infty} \frac{1}{x} \quad \quad \text{and}\quad \quad \lim_{x\to -\infty} \frac{1}{x}$$ By plug in some large numbers into $\frac{1}{x}$ we have
	$$	\begin{array}{c|c|c|c|c|c|c}
	-10000&-1000 & -100 ||\circ || 100 & 1000& 10000\\
	\hline
	-0.0001& -0.001 & -0.01  ||\circ || 0.01 & 0.001& 0.0001
	\end{array}
	$$
	We see that as $x$ is getting bigger and  positive the function $\frac{1}{x}$ is getting closer to  $0$. Thus, 
	$$\lim_{x\to +\infty} \frac{1}{x}=0.$$ Moreover,
	$$\lim_{x\to -\infty} \frac{1}{x}=0.$$
\end{Example}
\end{frame}

\begin{frame}
\begin{center}
{\bf \color{blue}Limit at Infinity}
\end{center}

\end{frame}

\begin{frame}
\begin{tcolorbox}[width=\textwidth,colback={red!5},title={},colbacktitle=yellow,coltitle=blue] 
\begin{definition}({\bf Informal limit at infinity.})
	We write 
	$$\lim_{x\to \infty} f(x)=L$$ when the value of the function $f(x)$ gets closer and closer to $L$ as we make $x$ larger and larger and positive.\\
	Similarly, we write 
	$$\lim_{x\to -\infty}f(x)=L$$ when the value of the function $f(x)$ gets closer and closer to $L$ as we make $x$ larger and larger and negative.
\end{definition}
\end{tcolorbox}
\end{frame}

\begin{frame}
\begin{columns}
	\begin{column}{.48\textwidth}
\begin{Example}
	Consider the graph of the function $f(x)$. 
	$$
	\includegraphics[width=.7\linewidth]{lim22}
	$$
	Then 
	$$\lim_{x\to \infty} f(x)=$$ $$ \lim_{x\to -\infty} f(x)=$$ 
\end{Example}
\end{column}



\begin{column}{.48\textwidth}
\begin{Example}
	Consider the graph of the function $g(x)$. 
	$$
	\includegraphics[width=.7\linewidth]{lim23}
	$$
	Then 
	$$\lim_{x\to \infty} g(x)=$$ $$ \lim_{x\to -\infty} g(x)=$$ 
\end{Example}
\end{column}
\end{columns}
\end{frame}


\begin{frame}
\begin{columns}
	\begin{column}{.48\textwidth}
		\begin{Example}
			Consider the graph of the function $f(x)$. 
			$$
			\includegraphics[width=.6\linewidth]{lim22}
			$$
			Then 
			$$\lim_{x\to \infty} f(x)=-2$$ $$ \lim_{x\to -\infty} f(x)=2$$ 
		\end{Example}
	\end{column}
	
	
	
	\begin{column}{.48\textwidth}
		\begin{Example}
			Consider the graph of the function $g(x)$. 
			$$
			\includegraphics[width=.6\linewidth]{lim23}
			$$
			Then 
			$$\lim_{x\to \infty} g(x)=-2$$ $$ \lim_{x\to -\infty} g(x)=+\infty$$ 
		\end{Example}
	\end{column}
\end{columns}
\end{frame}

\begin{frame}
\begin{center}
	{\bf\color{blue} Review of the third session}
\end{center}
\end{frame}

\begin{frame}{Review}
\begin{theorem}{\bf sandwich (or squeeze or pinch) } 
	Let $a\in \mathbb{R}$ and let $f,g,h$ be three functions so that 
	$$f(x) \leq g(x) \leq h(x)$$
	for all $x$ in an interval around $a$, except possibly at $x=a$. Then if 
	$$\lim_{x\to a}f(x)=\lim_{x\to a}h(x)=L$$
	then it is also the case that 
	$$\lim_{x\to a}g(x)=L.$$
\end{theorem}

\end{frame}
\begin{frame}

\begin{Example}
	Compute 
	$$\lim_{x\to 0} x^2\sin(\frac{\pi}{x})$$
\end{Example}
$$
\includegraphics[width=1\linewidth]{lim20}
$$
\end{frame}



\begin{frame}
\begin{tcolorbox}[width=\textwidth,colback={blue!10},title={},colbacktitle=yellow,coltitle=blue] 
\begin{theorem}
	Let $c\in \mathbb{R}$ then the following limits hold
	$$\lim_{x\to +\infty} c=c\quad \quad \quad \lim_{x\to -\infty} c=c$$
	$$\lim_{x\to +\infty} \frac{1}{x}=0$$
	$$\lim_{x\to -\infty} \frac{1}{x}=0.$$
\end{theorem}
\end{tcolorbox}
\end{frame}


\begin{frame}{\bf Outline For the Fourth Session}
\begin{itemize}
	\item Limit at Infinity
\end{itemize}
\end{frame}

\begin{frame}
\begin{center}
	{\bf \color{blue} Limit at Infinity}
\end{center}

\end{frame}

\begin{frame}
\begin{tcolorbox}[width=\textwidth,colback={blue!10},title={},colbacktitle=yellow,coltitle=blue] 
	\begin{theorem}
		Let $f(x)$ and $g(x)$ be two functions for which the limits 
		$$ \lim_{x\to \infty}f(x)=F\quad \quad \quad \lim_{x\to \infty}=G$$
		exist. Then the following limits hold
		$$
		\lim_{x\to \infty} (f(x)+g(x))=F\pm G
		$$
		$$
		\lim_{x\to \infty}f(x)g(x)=FG
		$$
		$$
		\lim_{x\to \infty}\frac{f(x)}{g(x)}=\frac{F}{G} \quad \text{provided}~G\neq 0
		$$
	\end{theorem}
\end{tcolorbox}
\end{frame}




\begin{frame}
\begin{tcolorbox}[width=\textwidth,colback={blue!10},title={},colbacktitle=yellow,coltitle=blue] 
		and for rational numbers $r$,
		$$\lim_{x\to \infty}(f(x))^r=F^r$$ provided that $f(x)^r$ is defined for all $x$.\\
		The analogous results hold for limits to $-\infty$.
\end{tcolorbox}
\end{frame}

\begin{frame}
	$
\includegraphics[width=.1\linewidth]{warning.jpeg}
$\\
\noindent {\bf \color{red}{Warning:}} Consider that 
$$\lim_{x\to +\infty}\frac{1}{x^{1/2}}=0$$ However, 
$$\lim_{x\to +\infty}\frac{1}{(-x)^{1/2}}$$  does not exist because $x^{1/2}$ is not defined for $x<0$.
\end{frame}

\begin{frame}{$f(x)=\frac{x^2-3x+4}{3x^2+8x+1}$}
$$
\includegraphics[width=1\linewidth]{contfun}
$$
\end{frame}

\begin{frame}{
	$$\sqrt{x^2}=|x|=\begin{cases}
x & x\geq 0\\
-x & x<0.
\end{cases}$$}
$$
\includegraphics[width=1\linewidth]{lim25}
$$
\end{frame}

\begin{frame}{
$$y=\frac{\sqrt{4x^2+1}}{5x-1}$$}
		$$
\includegraphics[width=1.2\linewidth]{lim26}
$$
\end{frame}

\begin{frame}

\begin{theorem}
	Let $a,c,H\in \mathbb{R}$ and let $f,g,h$ be functions defined in an interval around $a$ (but they need not be defined at $x=a$), so that 
	$$ \lim_{x\to a}f(x)=+\infty \quad \quad \lim_{x\to a}g(x)=+\infty \quad \quad \lim_{x\to a}h(x)=H$$
	\begin{enumerate}
		\item $$\lim_{x\to a}(f(x)+g(x))=$$
		\item $$\lim_{x\to a}(f(x)+h(x))=$$
		\item $$\lim_{x\to a}(f(x)-g(x))=$$
		\item $$\lim_{x\to a}(f(x)-h(x))=$$
	\end{enumerate}
\end{theorem}
\end{frame}

\begin{frame}
\begin{theorem}
	Let $a,c,H\in \mathbb{R}$ and let $f,g,h$ be functions defined in an interval around $a$ (but they need not be defined at $x=a$), so that 
	$$ \lim_{x\to a}f(x)=+\infty \quad \quad \lim_{x\to a}g(x)=+\infty \quad \quad \lim_{x\to a}h(x)=H$$
	\begin{enumerate}
		\item $$\lim_{x\to a}(f(x)+g(x))=+\infty.$$
		\item $$\lim_{x\to a}(f(x)+h(x))=+\infty.$$
		\item $$\lim_{x\to a}(f(x)-g(x))=\text{undetermined}.$$
		\item $$\lim_{x\to a}(f(x)-h(x))=+\infty.$$
	\end{enumerate}
\end{theorem}
\end{frame}

\begin{frame}
\begin{theorem}
	\begin{itemize}
\item[5.] $$\lim_{x\to a}cf(x)=\begin{cases}
~~~ & c>0\\
~~~& c=0\\
~~~ & c<0
\end{cases}$$
\item[6.] $$\lim(f(x).g(x))=$$
\item[7.] $$\lim_{x\to a} (f(x).h(x))
=\begin{cases}
~~~~ & H>0\\
 ~~~& H=0\\
~~~ & H<0
\end{cases}$$
\item[8.] $$\lim_{x\to a} \frac{h(x)}{f(x)}=$$
\end{itemize}
\end{theorem}
\end{frame}


\begin{frame}
\begin{theorem}
	\begin{itemize}
		\item[5.] $$\lim_{x\to a}cf(x)=\begin{cases}
		+\infty & c>0\\
		0 & c=0\\
		-\infty & c<0
		\end{cases}$$
		\item[6.] $$\lim(f(x).g(x))=+\infty.$$
		\item[7.] $$\lim_{x\to a} (f(x).h(x))
		=\begin{cases}
		+\infty & H>0\\
		\text{undetermined} & H=0\\
		-\infty & H<0
		\end{cases}$$
		\item[8.] $$\lim_{x\to a} \frac{h(x)}{f(x)}=0.$$
	\end{itemize}
\end{theorem}
\end{frame}

\begin{frame}

\begin{Example}
	Consider the following three functions:
	$$f(x)=x^{-2}\quad \quad g(x)=2x^{-2}\quad \quad h(x)=x^{-2}-1.$$
	Then 
	$$\lim_{x\to 0}f(x)=+\infty \quad \quad \lim_{x\to 0}g(x)=+\infty \quad \quad \lim_{x\to 0}h(x)=+\infty.$$
	Then 
	\begin{itemize}
		\item[1.] $$ \lim_{x\to 0} (f(x)-g(x))=$$
		\item[2.] $$ \lim_{x\to 0} (f(x)-h(x))=$$
		\item[3.] $$ \lim_{x\to 0} (g(x)-h(x))=$$
	\end{itemize}
\end{Example}
\end{frame}


\begin{frame}

\begin{Example}
	Consider the following three functions:
	$$f(x)=x^{-2}\quad \quad g(x)=2x^{-2}\quad \quad h(x)=x^{-2}-1.$$
	Then 
	$$\lim_{x\to 0}f(x)=+\infty \quad \quad \lim_{x\to 0}g(x)=+\infty \quad \quad \lim_{x\to 0}h(x)=+\infty.$$
	Then 
	\begin{itemize}
		\item[1.] $$ \lim_{x\to 0} (f(x)-g(x))=\lim_{x\to 0}x^{-2}=\infty$$
		\item[2.] $$ \lim_{x\to 0} (f(x)-h(x))=\lim_{x\to 0}(1)=1$$
		\item[3.] $$ \lim_{x\to 0} (g(x)-h(x))=\lim_{x\to 0}x^{-2}+1=\infty$$
	\end{itemize}
\end{Example}
\end{frame}


\begin{frame}{\bf Outline For the Session Five}
\begin{itemize}
	\item Limit at Infinity
	\item Continuity 
	\item Continuous from the left and from the right
	\item Arithmetic of continuity
	\item continuity of composites 
	\item Intermediate Value Theorem
\end{itemize}
\end{frame}

\begin{frame}
\begin{tcolorbox}[width=\textwidth,colback={green!10},title={},colbacktitle=yellow,coltitle=blue] 
\begin{example}
	$$ \lim_{x \to 0}\frac{1}{x^2} =\infty$$
$$	\begin{array}{c||c|c|c||c||c|c|c}
		x&-0.1&-0.01 & -0.001 & 0 & 0.001 & 0.01 & 0.1\\
		\hline
		&&&&&&&\\
	\frac{1}{x^2}	&100&10000&10^6& & 10^6 & 10000&100
	\end{array}
	$$
\end{example}
\end{tcolorbox}

Consider that if 
$$ \lim_{x\to a}f(x)=\infty~~~~  \lim_{x\to a}g(x)=\infty$$
Then  $$ \lim_{x\to a}(f(x)-g(x))=\text{undetermined}$$
\end{frame}

\begin{frame}
\begin{center}
{\bf \color{blue}{Continuity}}
\end{center}
\end{frame}


\begin{frame}
$$
\includegraphics[width=.4\linewidth]{x21}~	\includegraphics[width=.4\linewidth]{x3-2x+1}$$
$$
\includegraphics[width=.3\linewidth]{cont1}~	\includegraphics[width=.3\linewidth]{cont2}$$
\end{frame}

\begin{frame}{
	$$f(x)=\begin{cases}
x & x<1\\
x+2 & x\geq 1
\end{cases}$$}~~
\begin{mdframed}
	$$\text{jump discontinuity}\quad \quad 
	\includegraphics[width=.4\linewidth]{cont1}
	$$~~
\end{mdframed}
%\begin{itemize}
%	\item The function $f(x)$, at $x=1$ is ------------------- ; however, it is continuous form ------------ at $1$.
%	\item The function $f(x)$, on $[1,\infty)$ (for $x\geq 1$) is ------------------- .
%	\item The function $f(x)$, on $(-\infty,-1)$ is ------------------ .
%\end{itemize}
\end{frame}

%\begin{frame}{
%	$$f(x)=\begin{cases}
%	x & x<1\\
%	x+2 & x\geq 1
%	\end{cases}$$}
%\begin{mdframed}
%	$$\text{jump discontinuity}\quad \quad 
%	\includegraphics[width=.3\linewidth]{cont1}
%	$$
%\end{mdframed}
%%\begin{itemize}
%%	\item The function $f(x)$, at $x=1$ is {\color{red}{not continuous}}; however, it is continuous form {\color{red}{the right}} at $1$.
%%	\item The function $f(x)$, on $[1,\infty)$ (for $x\geq 1$) is {\color{red}{continuous}}.
%%	\item The function $f(x)$, on $(-1,-\infty)$ is {\color{red}{continuous}}.
%%\end{itemize}
%\end{frame}

\begin{frame}{
	$$g(x)=\begin{cases}
\frac{1}{x^2} & x\neq 0\\
0 & x=0
\end{cases}$$}
\begin{mdframed}
	$$\text{infinite discontinuity}\quad \quad 
	\includegraphics[width=.4\linewidth]{cont2}
	$$
\end{mdframed}
%\begin{itemize}
%	\item the function $g(x)$ is not continuous at $0$ ------------------- .
%	\item the function $g(x)$ is continuous at ----------------------- . 
%\end{itemize}
\end{frame}

%\begin{frame}{
%	$$g(x)=\begin{cases}
%	\frac{1}{x^2} & x\neq 0\\
%	0 & x=0
%	\end{cases}$$}
%\begin{mdframed}
%	$$\text{infinite discontinuity}\quad \quad 
%	\includegraphics[width=.3\linewidth]{cont2}
%	$$
%\end{mdframed}
%\begin{itemize}
%	\item the function $g(x)$ is not continuous at $0$ {\color{red}{neither form the right or the left.}}
%	\item the function $g(x)$ is continuous at {\color{red}{all points in $\mathbb{R}$ except $0$.}} 
%\end{itemize}
%\end{frame}

\begin{frame}{
$$h(x)=\begin{cases}
\frac{x^3-x^2}{x-1} & x\neq 1\\
0 & x=1
\end{cases}$$}
~~
\begin{mdframed}
	$$\text{removable discontinuity}\quad \quad 
	\includegraphics[width=.4\linewidth]{cont3}
	$$
\end{mdframed}
%\begin{itemize}
%	\item the function $h(x)$ is not continuous at $1$ neither form the right or the left.
%	\item the function $h(x)$ is continuous at all points in $\mathbb{R}$ except $1$. 
%\end{itemize}
\end{frame}


\begin{frame}
{\bf Outline - September 16, 2019}	
\begin{itemize}
\item {\bf Section 1.6:}
\begin{itemize}
\item Arithmetic of continuity
\item Continuity of composites 
\item Intermediate Value Theorem
\end{itemize}
\item {\bf Section 2.1:}
\begin{itemize}
	\item Revisiting tangent lines
\end{itemize}
\end{itemize}
\end{frame}


\begin{frame}
\begin{center}
	{\bf \color{blue}Arithmetic of continuity}
\end{center}
\end{frame}

\begin{frame}
\begin{tcolorbox}[width=\textwidth,colback={green!20},title={},colbacktitle=yellow,coltitle=blue] 
	\begin{theorem} ({\bf Arithmetic of continuity})
		Let $a,c\in \mathbb{R}$ and let $f(x)$ and $g(x)$ be functions that are continuous at $a$. Then the following functions are also continuous at $x=a$.
		\begin{itemize}
			\item $f(x)+g(x)$ and $f(x)-g(x)$,
			\item $cf(x)$ and $f(x)g(x)$, and 
			\item $\frac{f(x)}{g(x)}$ provided $g(a)\neq 0$. 
		\end{itemize}
	\end{theorem}
\end{tcolorbox}
\end{frame}

\begin{frame}
$$\includegraphics[width=1\linewidth]{sinexpanded}$$
\end{frame}

\begin{frame}
\begin{center}
{\bf \color{blue}Intermediate value theorem(IVT)}
\end{center}
\end{frame}



\begin{frame}

\begin{theorem}{\bf (Intermediate value theorem(IVT))} 
\end{theorem}

$$
\includegraphics[width=.7\linewidth]{IVT}
$$
\end{frame}

\begin{frame}{The existence not the uniqueness of $c$ in IVT}
$$
\includegraphics[width=.7\linewidth]{IVT4}
$$
\end{frame}

\begin{frame}{Not continuous functions at $[a,b]$ do not satisfy IVT}
$$
\includegraphics[width=1\linewidth]{IVT5}
$$
\end{frame}


\begin{frame}
\begin{center}
	{\bf \color{blue}Revisiting tangent lines}
\end{center}
\end{frame}

\begin{frame}{Revisiting tangent lines}
$$
\includegraphics[width=1\linewidth]{yx2}
$$
$$ \lim_{h\to 0}\frac{f(1+h)-f(1)}{h} \leftarrow \text{~ slope of the tangent line at $x=1$}$$
\end{frame}



%
%\begin{frame}
%
%\begin{tcolorbox}[width=\textwidth,colback={blue!10},title={},colbacktitle=yellow,coltitle=blue] 
%\begin{theorem}
%	Let $a,c\in \mathbb{R}$ and let $f(x)$ and $g(x)$ be functions that are continuous at $a$. Then the following functions are also continuous at $x=a$.
%	\begin{itemize}
%		\item $f(x)+g(x)$ and $f(x)-g(x)$,
%		\item $cf(x)$ and $f(x)g(x)$, and 
%		\item $\frac{f(x)}{g(x)}$ provided $g(a)\neq 0$. 
%	\end{itemize}
%\end{theorem}
%\end{tcolorbox}
%\end{frame}




%	
%	\begin{frame}
%	\frametitle{\bf Limits}
%What does this mean  $$\underset{x\rightarrow a}{\rm lim}f(x)=L?$$
%
%The "limit" appears when we want to
%\begin{itemize}
%	\item find the tangent to a curve; or 
%	\item find the velocity of an object.
%\end{itemize} 
%
%\end{frame}
%
%\begin{frame}
%\frametitle{\bf Tangent line}
%$$
%\includegraphics[width=1\linewidth]{tangent}
%$$
%$$
%\includegraphics[width=1\linewidth]{tangent2}
%$$
%\end{frame}
%
%\begin{frame}
%\frametitle{{\bf The equation of a tangent line: Example}}
%
%	Find the tangent line to the curve $y=x^2$ that passes through $P=(1,1)$.
%$$
%\includegraphics[width=.7\linewidth]{yx2}
%$$
%The slope for the green line (secant line)
%$$
%m=\frac{y_2-y_1}{x_2-x_1}=\frac{ (1+h)^2-1^2 }{(1+h)-1}=\frac{1+2h+h^2-1}{h}=\frac{h(h+2)}{h}=2+h
%$$
%\end{frame}
%\begin{frame}
%$$
%\begin{array}{|c|c|}
%\hline\\
%h & m=\frac{ (1+h)^2-1^2 }{(1+h)-1}\\
%\hline\\
%0.1 & 2.1\\
%0.01 & 2.01\\
%0.001 & 2.001\\
%\hline
%\end{array}
%$$
%We can write this more {\color{blue}mathematically} as
%\begin{center}
%\noindent\fbox{%
%	\parbox{4cm}{%
%		$$ \underset{h\rightarrow 0}{\rm lim} \frac{(1+h)^2-1}{(1+h)-1}=2$$
%	}%
%}
%\end{center}
%{\bf Read:}  {\it The limit of $\frac{(1+h)^2-1}{(1+h)-1}$ as $h$ approaches $0$ is $2$}.
%\end{frame}
%
%
%\begin{frame}
%\frametitle{\bf Velocity: Example}
%Someone drops a ball at the top of a very tall building. 
%\begin{itemize}
%	\item[--] Let $t$ be elapsed time measured in second
%	\item[--] $S(t)$ be the distance the ball has fallen in meters
%	\item[--] What is $S(0)$? $S(0)=0$.
%	\item[--] ({\bf Galileo}) $S(t)=4.9t^2$.
%\end{itemize}
%\noindent {\bf Question:} How fast the ball is fallen after $1$ second, that is,  find $v(1)$, the velocity at $t=1$ ?
%\end{frame}
%
%\begin{frame}
%	$$
%\includegraphics[width=1\linewidth]{velocity1}
%$$
%$$
%\text{average velocity}=\frac{\text{difference in position}}{\text{difference in time}}=\frac{S(1.1)-S(1)}{1.1-1}
%$$
%$$=\frac{4.9(1.1)^2-4.9}{0.1}=10.29$$
%
%\end{frame}
%\begin{frame}
%\frametitle{Average velocity when $(t_2=1+h)$ and $(t_1=1)$}
%$$
%\text{average velocity when $(t_2=1+h)$ and $(t_1=1)$}$$
%$$=\frac{S(1+h)-S(1)}{h}=\frac{4.9(1+h)^2-4.9}{h}=4.9(2+h).
%$$
%
%$$
%\begin{array}{|c|c|}
%\hline\\
%\text{time window} & \text{average velocity}\\
%\hline\\
%1\leq t\leq 1.1 & 10.29\\
%1\leq t\leq 1.01 & 9.84\\
%1\leq t\leq 1.01  & 9.8049\\
%1\leq t\leq 1.001  & 9.80049\\
%\hline
%\end{array}
%$$
%So we can write 
%$$v(1)= \underset{h\rightarrow 0}{\rm lim} \frac{S(1+h)-S(1)}{h}=9.8.$$
%\end{frame}
%






\end{document}