\message{ !name(Math4140notes.tex)}
\documentclass[12pt,a4paper]{book}

% PAGELAYOUT
\headheight=8pt    \topmargin=0pt
\textheight=624pt \textwidth=432pt
\oddsidemargin=18pt \evensidemargin=18pt

% PACKAGES
\usepackage{amsthm}
 \usepackage{amsmath}
 \usepackage[margin=1in]{geometry}
 \usepackage{amsfonts}
 \usepackage{graphicx}
 \usepackage{mdframed}
 \usepackage{dirtytalk}
 \usepackage{blindtext}
 \usepackage{tcolorbox}
 \usepackage{graphicx}
\usepackage{hyperref}
\usepackage{pgfplots}

\pgfplotsset{compat=1.10}
\usepgfplotslibrary{fillbetween}

% THEOREMS AND STRUCTURES
\newtheorem{theorem}{Theorem}[section]
\newtheorem{corollary}[theorem]{Corollary}
\newtheorem{lemma}[theorem]{Lemma}
\newtheorem{Example}[theorem]{Example}
\newtheorem{proposition}[theorem]{Proposition}
\numberwithin{equation}{section}
\newtheorem*{definition}{Definition}
\newtheorem*{solution}{{\bf Solution}}
\newtheorem*{remark}{Remark}

\definecolor{moonstoneblue}{rgb}{0.45, 0.66, 0.76}

\begin{document}

\message{ !name(Math100.tex) !offset(-3) }


%% TITLE, AUTHOR'S NAME AND ADDRESS
 \title{MATH100: Differential Calculus with Application to Physical Sciences and Engineering}
 \author{University of British Columbia\\ \\Farid Aliniaeifard}


\maketitle

% TABLE OF CONTENTS
\tableofcontents



\chapter{Limits}
What does this mean  $$\underset{x\rightarrow a}{\rm lim}f(x)=L?$$

The "limit" appears when we want to
\begin{itemize}
	\item find the tangent to a curve; or 
	\item find the velocity of an object.
\end{itemize} 
\section{  Tangent line}




%So what is a tangent line to a curve? We don't give a formal definition of the tangent to a curve, however through some examples we clarify this definition. 
%\newpage

$$
	\includegraphics[width=1\linewidth]{tangent}
$$
$$
\includegraphics[width=1\linewidth]{tangent2}
$$
The {\bf tangent line to a curve $y=f(x)$ at a point $P$} (if exists) is a line $L$ that there is a neighborhood for $P$ such that in that  neighborhood the line $L$ touches (does not cross) the curve $y=f(x)$ only at $P$ (and not other points in that neighborhood).


\subsection*{The equation of a line}
\begin{itemize}
\item[--] The formula for a line that passes though $(x_1,y_1)$ with slope $m$ is 
$$ y=y_1+m(x-x_1).$$
\item[--] Given two points $(x_1,y_1)$ and $(x_2,y_2)$ on a line, then the slope of the line is $$m=\frac{y_2-y_1}{x_2-x_1},$$
and the formula for the line then is 
$$ y=y_1+m(x-x_1).$$
\end{itemize}
	\begin{Example} 
Find the equation of the line with slope $-3$ that passes through $(1,2)$. 
\begin{solution}
	The equation of the line is $$y=2+(-3)(x-1), \text{~so~}y=5-3x.$$	
\end{solution}		
		\end{Example} 
\begin{Example}
	Find the equation of the line that passes through $(1,2)$ and $(2,-1)$.
\end{Example}
\begin{solution}
	First we find the slope which is 
	$$\frac{-1-2}{2-1}=-3.$$ Then the equation of the line is 
$$y=2+(-3)(x-1), \text{~so~}y=5-3x.$$	
\end{solution}




\noindent {\bf The equation of a tangent line:} Given a curve $y=f(x)$ and a point $P$ on the curve, how to find the slope of the tangent to a curve at $P$:  let do this through an example.

\begin{Example}
	Find the tangent line to the curve $y=x^2$ that passes through $P=(1,1)$.
\end{Example}
$$
\includegraphics[width=1\linewidth]{yx2}
$$
$$
\includegraphics[width=1\linewidth]{yx22}
$$

%$$
%	\includegraphics[width=1\linewidth]{yx2}
%$$
So we want to find the slope the line that passes through the points $(x_1,y_1)=(1,1)$ and $(x_2,y_2)=(1+h,(1+h)^2)$. The slope then is 
$$
m=\frac{y_2-y_1}{x_2-x_1}=\frac{ (1+h)^2-1^2 }{(1+h)-1}=\frac{1+2h+h^2-1}{h}=\frac{h(h+2)}{h}=2+h
$$
$$
\begin{array}{|c|c|}
\hline\\
h & m=\frac{ (1+h)^2-1^2 }{(1+h)-1}\\
\hline\\
0.1 & 2.1\\
0.01 & 2.01\\
0.001 & 2.001\\
\hline
\end{array}
$$
When $h$ gets smaller and smaller, the slope will be closer and closer to the slope of the tangent line to $y=x^2$ at $(1,1)$, which the slope will be closer and closer to $2$, we can write this more mathematically as 
$$ \underset{h\rightarrow 0}{\rm lim} \frac{(1+h)^2-1}{(1+h)-1}=2$$
{\bf Read:}  {\it the limit of $\frac{(1+h)^2-1}{(1+h)-1}$ as $h$ approaches $0$ is $2$}.
\\
Tangent line is
$$y=1+2(x-1)=2x-1.$$

\section{Velocity}

\noindent\begin{minipage}{0.5\textwidth} 
	\begin{itemize}
		\item[--] Let $t$ be elapsed time measured in second
		\item[--] $S(t)$ be the distance the ball has fallen in meters
		\item[--] What is $S(0)$? $S(0)=0$.
		\item[--] ({\bf Galileo}) $S(t)=4.9t^2$.
	\end{itemize}

\end{minipage}
\hfill
\begin{minipage}{0.5\textwidth}\raggedleft
$$	\includegraphics[width=1\linewidth]{fall.jpg} $$
\end{minipage} 
\\
\\
\noindent {\bf Question:} How fast the ball is fallen after $1$ second, that is,  find $v(1)$, the velocity at $t=1$ ?
$$
\text{average velocity}=\frac{\text{difference in position}}{\text{difference in time}}=\frac{S(t_2)-S(t_1)}{t_2-t_1}.
$$
To answer the question we should find the average velocity of the falling ball between $(1+h)$ and $1$. So,
$$
\text{average velocity when $(t_2=1+h)$ and $(t_1=1)$}$$
$$=\frac{S(1+h)-S(1)}{h}=\frac{4.9(1+h)^2-4.9}{h}=4.9(2+h).
$$
$$
\includegraphics[width=1\linewidth]{velocity.jpg}
$$
$$
\begin{array}{|c|c|}
\hline\\
\text{time window} & \text{average velocity}\\
\hline\\
1\leq t\leq 1.1 & 10.29\\
1\leq t\leq 1.01 & 9.84\\
1\leq t\leq 1.01  & 9.8049\\
1\leq t\leq 1.001  & 9.80049\\
\hline
\end{array}
$$
So we can write 
$$v(1)= \underset{h\rightarrow 0}{\rm lim} \frac{S(1+h)-S(1)}{h}=9.8.$$
	More generally:
 \begin{mdframed}
We define the instantaneous velocity at time $t=a$ to be the limit 
	$$v(a)= \underset{h\rightarrow 0}{\rm lim} \frac{S(a+h)-S(a)}{h}$$
 \end{mdframed}

\newpage 
\section{The limit of a function}

To arrive at the definition of limit, we start with a very simple example.
\begin{Example}
	Consider the following function
	$$
	f(x)=\begin{cases}
	2x & x<3\\
	9 & x=3\\
	2x & x>3
	\end{cases}
	$$
	$$
	\includegraphics[width=1\linewidth]{lim1}
	$$
	If we plug in some numbers very close to $3$ (but not exactly 3) into the function we see
		$$
	\includegraphics[width=.6\linewidth]{lim2}
	$$
	So as $x$ moves closer and closer to $3$, without being exactly $3$, we see that the function moves closer and closer to $6$. We can then write this as 
	$$\underset{x\rightarrow 3}{\rm lim}f(x)=6.$$ \end{Example}
	\begin{mdframed}
	\begin{definition}
		{\bf (Informal definition of limit)} We write 
		$$\underset{x\rightarrow a}{\rm lim}f(x)=L.$$
		if the value of the function $f(x)$ is sure to be arbitrary close to $L$ whenever the value of $x$ is close enough to $a$, without being exactly $a$.
	\end{definition}
	\end{mdframed}

\begin{Example}
	Let $f(x)=\frac{x-2}{x^2+x-6}$ and find its limit as $x\rightarrow 2$.  
\end{Example}
\begin{solution}
	We want to find
		$$\underset{x\rightarrow 2}{\rm lim}\frac{x-2}{x^2+x-6}.$$
		{\bf Important point:} if we we compute $f(2)$, then we have $\frac{0}{0}$ which is undefined.
		
		Again we plug in numbers close to $2$ and we have 
	$$
\includegraphics[width=.7\linewidth]{lim3}
$$
		So
		$$\underset{x\rightarrow 2}{\rm lim}\frac{x-2}{x^2+x-6}=2.$$
		
\end{solution}
\begin{Example} Consider the following function $f (x) = sin(\pi/x)$. Find the limit as  $x\rightarrow 0$ of $f(x)$.
\end{Example}

\begin{solution}  When $x$ is getting closer and closer to $0$, it
oscillates faster and faster. Since the function does not approach
a single number as we bring $x$ closer and closer to zero, the limit does not exist. Thus, 
$$\underset{x\rightarrow 0}{\rm lim}sin(\pi/x)=DNE$$
		$$
	\includegraphics[width=1\linewidth]{lim4}
	$$
\end{solution}

\begin{Example}
	Consider the function 
	$$f(x)=
	\begin{cases}
	x & x<2\\
	-1 & x=2\\
	x+3 & x>2
	\end{cases}
	$$
	Find $$\underset{x\rightarrow 2}{\rm lim}~f(x).$$
\end{Example}

\begin{solution}
		$$
	\includegraphics[width=1\linewidth]{lim8}
	$$
	Let again plug in some numbers close to $2$ (but not exactly $2$)
		$$
	\includegraphics[width=.7\linewidth]{lim7}
	$$
	Now when we approach from below (or left), we seem to be getting closer to $2$ {\color{red}{ $(\underset{x\rightarrow 2^{-}}{\rm lim}f(x)=2)$, }} but when we approach from above (or right) we seem to be getting closer to $5$ {\color{red}{ $(\underset{x\rightarrow 2^{+}}{\rm lim}f(x)=5)$ }}. Since we are not approaching the same number the limit does not exists. 
	$$\underset{x\rightarrow 2}{\rm lim}f(x)=DNE$$
	
\end{solution}

\begin{mdframed}
\begin{definition}({\bf Informal definition of one-sided limits})
	We write 
	$$
	\underset{x\rightarrow a^-}{\rm lim}f(x)=K
	$$ when the value of $f(x)$ gets closer and closer to $K$ when $x<a$ and $x$ moves closer and closer to $a$. Since the $x$-values are always less than $a$, we say that $x$ approaches $a$ from below (or left). This is also often called the left-hand limit since the $x$-values lie to the left of $a$ on a sketch of the graph.
	
	We similarly write 
	$$\underset{x\rightarrow a^{+}}{\rm lim}f(x)=L$$ when the values of $f(x)$ gets closer and closer to $L$ when $x>a$ and $x$ moves closer and closer to $a$. For similar reason we say that $x$ approaches $a$ from above, and sometimes to this as the the right-hand limit. 
\end{definition}
\end{mdframed}

\begin{mdframed}
\begin{theorem}
	$$\underset{x\rightarrow a}{\rm lim}f(x)=L\quad \text{if and only if} \quad \underset{x\rightarrow a^-}{\rm lim}f(x)=L \quad \text{and} \quad \underset{x\rightarrow a^+}{\rm lim}f(x)=L$$
\end{theorem}
\end{mdframed}

\begin{itemize}
	\item If the limit of $f(x)$ as $x$ approaches $a$ exists and is equal to $L$, then both the left-hand
and right-hand limits exist and are equal to $L$. 
\item 
 If the left-hand and right-hand limits as $x$ 
 approaches a exist and are equal, then the
limit as $x$ approaches a exists and is equal to the one-sided limits.
\end{itemize}
Contrapositive of the above argument says
\begin{itemize}
\item If either of the left-hand and right-hand limits as $x$ approaches a fail to exist, or if
they both exist but are different, then the limit as $x$ approaches a does not exist.
AND,
\item If the limit as $x$ approaches a does not exist, then the left-hand and right-hand limits
are either different or at least one of them does not exist.
\end{itemize}

\begin{Example} Consider the graph of the function $f(x)$.
		$$
	\includegraphics[width=.5\linewidth]{lim9}
	$$
	Then 
	$$
	\underset{x\rightarrow 1^-}{\rm lim}f(x)=2  \quad \quad 
	\underset{x\rightarrow 1^+}{\rm lim}f(x)= 2 \quad \quad  	\underset{x\rightarrow 1}{\rm lim}f(x)=2
	$$
\end{Example}
\begin{Example} Consider the graph of the function $g(x)$.
	$$
	\includegraphics[width=.5\linewidth]{lim10}
	$$
	Then 
	$$
	\underset{t\rightarrow 1^-}{\rm lim}g(t)=2  \quad \quad 
	\underset{t\rightarrow 1^+}{\rm lim}g(t)= -2 \quad \quad  	\underset{t\rightarrow 1}{\rm lim}g(t)=DNE
	$$
\end{Example}
In the following example even though the limit doesn't exists when $x$ approaches $a$, we can say more. 
\begin{Example}
	Consider the graph for the function $f(x)$.
		$$
	\includegraphics[width=.4\linewidth]{lim11}
	$$
	$$
%	\underset{x\rightarrow a^-}{\rm lim}f(x)=+\infty  \quad \quad 
%	\underset{x\rightarrow a^+}{\rm lim}f(x)= +\infty \quad \quad
	  	\underset{x\rightarrow a}{\rm lim}f(x)=+\infty
	$$
\end{Example}

		
	\begin{Example}
		Consider the graph for the function $g(x)$.
		$$
		\includegraphics[width=.4\linewidth]{lim12}
		$$
		$$
	%	\underset{x\rightarrow a^-}{\rm lim}g(x)=-\infty  \quad \quad 
	%	\underset{x\rightarrow a^+}{\rm lim}g(x)= -\infty \quad \quad  
			\underset{x\rightarrow a}{\rm lim}g(x)=-\infty
		$$
	\end{Example}

\begin{mdframed}
	\begin{definition}
		We write 
		$$\underset{x\rightarrow a}{\rm lim}f(x)=+\infty$$
		when the value of the function $f(x)$ becomes arbitrarily large and positive as $x$ gets closer and closer to $a$, without being exactly $a$.
		\\
		Similarly, we write 
		$$\underset{x\rightarrow a}{\rm lim}f(x)=-\infty$$
		when the value of the function $f(x)$ becomes arbitrarily large and negative as $x$ gets closer and closer to $a$, without being exactly $a$.
	\end{definition}
\end{mdframed}
\begin{Example}
	$$\underset{x\rightarrow 0}{\rm lim}\frac{1}{x^2}=+\infty \quad  \quad \underset{x\rightarrow 0}{\rm lim}-\frac{1}{x^2}=-\infty$$
\end{Example}

\noindent {\bf Important Point:} Do not think of \say{$+\infty$} and \say{$-\infty$} in these statements as numbers. When we write $\underset{x\rightarrow a}{\rm lim}f(x)=+\infty$, it says \say{the function $f(x)$ becomes arbitrary large as $x$ approaches $a$}.
	
	\begin{Example}
	Consider the graph for the function $h(x)$.
	$$
	\includegraphics[width=.4\linewidth]{lim13}
	$$
	$$
	\underset{x\rightarrow a^-}{\rm lim}h(x)=+\infty  \quad \quad 
	\underset{x\rightarrow a^+}{\rm lim}h(x)= 3 \quad \quad  	\underset{x\rightarrow a}{\rm lim}h(x)=DNE
	$$
\end{Example}	
	\begin{Example}
	Consider the graph for the function $s(x)$.
	$$
	\includegraphics[width=.4\linewidth]{lim14}
	$$
	$$
	\underset{x\rightarrow a^-}{\rm lim}s(x)=3  \quad \quad 
	\underset{x\rightarrow a^+}{\rm lim}s(x)=-\infty \quad \quad  	\underset{x\rightarrow a}{\rm lim}s(x)=DNE
	$$
\end{Example}

\begin{mdframed}
\begin{definition}
	We write $$\underset{x\rightarrow a^+}{\rm lim}f(x)=+\infty$$ when the value of the function $f(x)$ becomes arbitrarily large and positive as $x$ gets closer and closer to $a$ from above (equivalently, from right), without being exactly $a$.\\
	Similarly, we write 
	$$\underset{x\rightarrow a^+}{\rm lim}f(x)=-\infty$$ when the values of the function $f(x)$ becomes arbitrarily large and negative as $x$ gets closer and closer to $a$ from above (equivalently, from right), without being exactly $a$. 
	\\
	The notation 
	$$ \underset{x\rightarrow a^-}{\rm lim}f(x)=+\infty \quad \text{and} \quad\underset{x\rightarrow a^-}{\rm lim}f(x)=-\infty $$ has a similar meaning except that limits are approached form below (from left).
\end{definition}
\end{mdframed}



\begin{Example}
	Consider the function $$g(x)=\frac{1}{sin(x)}.$$ Find the one-side limits of this function as $x\rightarrow \pi$.
\end{Example}
	$$
\includegraphics[width=1\linewidth]{lim15}
$$

	$$
\includegraphics[width=1\linewidth]{lim17}
$$



\newpage
\section{Calculating Limits with Limit Laws} 

\begin{mdframed}
	\begin{theorem}
	Let $a,c\in \mathbb{R}$. The following two limits hold 
	$$  	\underset{x\rightarrow a}{\rm lim}c=c \quad \quad 	\underset{x\rightarrow a}{\rm lim}x=a $$
	\end{theorem}
	
\end{mdframed}




\begin{tcolorbox}[width=\textwidth,colback={blue!10},title={},colbacktitle=yellow,coltitle=blue]    
\begin{theorem}(Arithmetic of Limits)
	Let $a,c\in \mathbb{R}$, let $f(x)$ and $g(x)$ be defined for all $x$'s that lie in some interval about $a$ (but $f$ and $g$ need not to be defined exactly at $a$).
	$$\underset{x\rightarrow a}{\rm lim}f(x)=F \quad \quad \underset{x\rightarrow a}{\rm lim}g(x)=G$$ 
	exists with $F,G\in \mathbb{R}$. Then the following limits hold
	\begin{itemize}
		\item 	$ \underset{x\rightarrow a}{\rm lim}(f(x)+g(x))=F+G$--limit of the sum is the sum of the limits.
		\item $ \underset{x\rightarrow a}{\rm lim}(f(x)-g(x))=F-G$--limit of the difference is the difference of the limits.
		\item $\underset{x\rightarrow a}{\rm lim}cf(x)=cF$.
		\item $ \underset{x\rightarrow a}{\rm lim}(f(x).g(x))=F.G$--limit of the product is the product of the limits.
		\item If $G\neq 0$ then $\underset{x\rightarrow a}{\rm lim}\frac{f(x)}{g(x)}=\frac{F}{G}$.
	\end{itemize}
\end{theorem}
\end{tcolorbox} 

\begin{Example}
	Given 
	$$\underset{x\rightarrow 1}{\rm lim}f(x)=3\quad \text{and} \quad \underset{x\rightarrow 1}{\rm lim}g(x)=2 $$
	We have 
	 $$\underset{x\rightarrow 1}{\rm lim}3f(x)=3\times \lim_{x\to 1}f(x)=3\times 3=9.$$
		$$\underset{x\rightarrow 1}{\rm lim}3f(x)-g(x)=3\times \lim_{x\to 1}f(x)-\lim_{x\to 1}g(x)=3\times 3- 2=7.$$
		 $$\underset{x\rightarrow 1}{\rm lim}f(x)g(x)=\lim_{x\to 1}f(x).\lim_{x\to 1}g(x)=3\times 2=6.$$
		 $$\underset{x\rightarrow 1}{\rm lim} \frac{f(x)}{f(x)-g(x)}=\frac{\lim_{x\to 1}f(x)}{\lim_{x\to 1}f(x)-\lim_{x\to 1}g(x)}=\frac{3}{3-2}=3.$$
\end{Example}
\begin{Example} 
	
		$$\lim_{x\to 3}4x^2-1=4\times \lim_{x\to 3}x^2-\lim_{x\to 3}1=35.$$
		 $$\lim_{x\to 2}\frac{x}{x-1}=\frac{\lim_{x\to 2}x}{\lim_{x\to 2}x-\lim_{x\to 1}1}=\frac{2}{2-1}=2.$$
	
\end{Example}
Consider that we apply the theorem  Arithmetic of Limits to compute the limit of a ratio if the limit of denominator is not zero. 
{\bf What will happen if the limit of denominator is zero:}
\begin{itemize}
	\item[--] the limit does not exist, eg. 
	$$  \lim_{x\to 0} \frac{x}{x^2}=\lim_{x\to 0} \frac{1}{x}=DNE$$
	\item[--] the limit is $\pm \infty$, eg. 
	$$\lim_{x\to 0} \frac{x^2}{x^4}=\lim_{x\to 0} \frac{1}{x^2}=+\infty \quad \quad \text{or} \quad \quad \lim_{x\to 0} \frac{-x^2}{x^4}=\lim_{x\to 0} \frac{-1}{x^2}=-\infty.$$
	\item[--] the limit is $0$, eg. $$\lim_{x\to 0} \frac{x^2}{x}=\lim_{x\to 0} x=0.$$
	\item[--] the limit exists and it nonzero, eg. $$\lim_{x\to 0} \frac{x}{x}=1.$$
\end{itemize}

\begin{tcolorbox}[width=\textwidth,colback={blue!10},title={},colbacktitle=yellow,coltitle=blue]  
\begin{theorem}
	Let $n$ be a positive integer, let $a\in R$ and let $f$ be a function so that
	$$\lim_{x\to a}f(x)=F$$ for some real number $F$. Then the following holds
	$$ \lim_{x\to a}(f(x))^n=\left( \lim_{x\to a} f(x) \right)^n=F^n$$ so that the limit of a power is the power of the limit. Similarly, if 
	\begin{itemize}
		\item $n$ is an even number and $F>0$, or 
		\item $n$ is an odd number and $F$ is any real number 
	\end{itemize}
then 
$$ \lim_{x\to a}(f(x))^{1/n}=\left( \lim_{x\to a}f(x) \right)^{1/n}=F^{1/n}.$$
	\end{theorem}
\end{tcolorbox}

\begin{Example}
	$$\lim_{x\to 4}x^{1/2}=4^{1/2}=2.$$
	$$\lim_{x\to 4}(-x)^{1/2}=-4^{1/2}=\text{not a real number}.$$
	$$ \lim_{x\to 2} (4x^2-3)^{1/3}= (4(2)^2-3)^{1/3}=13^{1/3}$$
\end{Example}

\begin{Example} Compute the following limits.
\begin{enumerate}	
	\item $\lim_{x\to 2} \frac{x^3-x^2}{x-1}$
\item $\lim_{x\to 1} \frac{x^3-x^2}{x-1}$
\end{enumerate}
\end{Example}

\begin{solution}
	1. $\lim_{x\to 2} \frac{x^3-x^2}{x-1}=4.$\\
	2. Consider that $\lim_{x\to 1} {x^3-x^2}=0$ and $\lim_{x\to 1} {x-1}=0$. However, 
	$$ \frac{x^3-x^2}{x-1}=\frac{x^2(x-1)}{x-1},$$ thus
	$$ \frac{x^3-x^2}{x-1}=\begin{cases}
	x^2 & x\neq 1\\
	\text{undefined} & x=1.
	\end{cases}
	$$
		$$
	\includegraphics[width=1\linewidth]{lim18}
	$$
	And so
	$$ \lim_{x\to 1} \frac{x^3-x^2}{x-1}=\lim_{x\to 1} {x^2}=1. $$
\end{solution}
The reasoning in the above example can be made more general:
\begin{theorem}
	If $f(x)=g(x)$ except when $x=a$ then $$\lim_{x\to a} f(x)=\lim_{x\to a} g(x)$$ provided the limit of $g$ exists. 
\end{theorem}
We mostly use the above theorem when we end up with $\frac{0}{0}$.

\begin{Example}
	Compute 
	$$\lim_{h\to 0} \frac{(1+h)^2-1}{h}.$$
\end{Example}
\begin{solution}
	Note that 
	$$\frac{(1+h)^2-1}{h}=\frac{1+2h+h^2-1}{h}=\frac{h(2+h)}{h}.$$ Thus,
	$$\frac{(1+h)^2-1}{h}=\begin{cases}
	2+h & h\neq 0\\
	\text{undefined} & h=0.
	\end{cases}$$
	And so $$\lim_{h\to 0}\frac{(1+h)^2-1}{h}=\lim_{h\to 0}2+h=2.$$
\end{solution}
We now present a slightly harder example.
\begin{Example}
	Compute the limit 
	$$\lim_{x\to 0}\frac{x}{\sqrt{1+x}-1}.$$
\end{Example}

\begin{solution}
	Both the limits of the numerator and denominator as $x \to 0$ are $0$, so we cannot use  the Theorem  Arithmetic of limits. We now can simply multiply the numerator and denominator by the conjugation of $\sqrt{1+x}-1$, that is, $\sqrt{1+x}+1$. We have
		$$
	\includegraphics[width=1\linewidth]{lim19}
	$$
	
\end{solution}
Before we move to the next section and study the limits at infinity, we have one more theorem to state. 
\begin{Example}
	Compute 
	$$\lim_{x\to 0} x^2\sin(\frac{\pi}{x})$$
\end{Example}
	$$
\includegraphics[width=1\linewidth]{lim20}
$$

\begin{solution}
	It is not possible to simply use the theorem Arithmetic of Limits since the limit of $\sin(\frac{\pi}{x})$ as $x\to 0$ does not exist.  Since $-1\leq \sin (\theta)\leq 1$ for all real numbers $\theta$, we have 
	$$-1\leq \sin(\frac{\pi}{x})\leq 1 \quad \quad \text{for all~}x\neq 0$$
	Multiplying the above by $x^2$ we see that 
	$$-x^2\leq x^2\sin(\frac{\pi}{x})\leq x^2 \quad \quad \text{for all~}x\neq 0.$$
	Since  $$\lim_{x\to 0}x^2=\lim_{x\to 0}(-x^2)=0$$ by the {\bf sandwich (or squeeze or pinch) theorem} (look at below for the sandwich theorem) we have 
	$$\lim_{x\to 0} x^2\sin(\frac{\pi}{x})=0.$$
\end{solution}

\begin{theorem}({\bf sandwich (or squeeze or pinch) theorem} )
	Let $a\in \mathbb{R}$ and let $f,g,h$ be three functions so that 
	$$f(x) \leq g(x) \leq h(x)$$
	for all $x$ in an interval around $a$, except possibly at $x=a$. Then if 
	$$\lim_{x\to a}f(x)=\lim_{x\to a}h(x)=L$$
	then it is also the case that 
	$$\lim_{x\to a}g(x)=L.$$
\end{theorem}

\begin{Example}
	Let $f(x)$ be a function such that $1\leq f(x)\leq x^2-2x+2$. What is 
	$$\lim_{x\to 1}f(x)?$$
\end{Example}

\begin{solution}
	Consider that 
	$$\lim_{x\to 1}x=1\quad \quad \quad \text{and}\quad \quad \quad \lim_{x\to 1}x^2-2x+2=1.$$
	Therefore, by the sandwich/pinch/squeeze theorem 
	$$\lim_{x\to 1}f(x)=1.$$
\end{solution}

\section{Limits at Infinity} 
\begin{Example}
	We want to compute 
	$$\lim_{x\to +\infty} \frac{1}{x} \quad \quad \text{and}\quad \quad \lim_{x\to -\infty} \frac{1}{x}$$ By plug in some large numbers into $\frac{1}{x}$ we have
$$	\begin{array}{c|c|c|c|c|c|c}
		-10000&-1000 & -100 ||\circ || 100 & 1000& 10000\\
		\hline
		-0.0001& 0.001 & -0.01  ||\circ || 0.01 & 0.001& 0.0001
	\end{array}
	$$
	We see that as $x$ is getting bigger and  positive the function $\frac{1}{x}$ is getting closer to  $0$. Thus, 
	$$\lim_{x\to +\infty} \frac{1}{x}=0.$$ Moreover,
	$$\lim_{x\to -\infty} \frac{1}{x}=0.$$
\end{Example}

\begin{definition}({\bf Informal limit at infinity.})
	We write 
	$$\lim_{x\to \infty} f(x)=L$$ when the value of the function $f(x)$ gets closer and closer to $L$ as we make $x$ larger and larger and positive.\\
	Similarly, we write 
	$$\lim_{x\to -\infty}f(x)=L$$ when the value of the function $f(x)$ gets closer and closer to $L$ as we make $x$ larger and larger and negative.
\end{definition}

\begin{Example}
	Consider the graph of the function $f(x)$. 
		$$
	\includegraphics[width=.6\linewidth]{lim22}
	$$
	Then 
	$$\lim_{x\to \infty} f(x)=-2\quad \quad \text{and}\quad \quad \lim_{x\to -\infty} f(x)=2$$ 
\end{Example}

\begin{Example}
	Consider the graph of the function $g(x)$. 
	$$
	\includegraphics[width=.6\linewidth]{lim23}
	$$
	Then 
	$$\lim_{x\to \infty} g(x)=-2\quad \quad \text{and}\quad \quad \lim_{x\to -\infty} g(x)=+\infty$$ 
\end{Example}
Same as usual we start with two very simple building blocks and build other limits from them.
\begin{theorem}
	Let $c\in \mathbb{R}$ then the following limits hold
	$$\lim_{x\to +\infty} c=c\quad \quad \quad \lim_{x\to -\infty} c=c$$
		$$\lim_{x\to +\infty} \frac{1}{x}=0\quad \quad \quad \lim_{x\to -\infty} \frac{1}{x}=0.$$
\end{theorem}
\begin{tcolorbox}[width=\textwidth,colback={blue!10},title={},colbacktitle=yellow,coltitle=blue] 
\begin{theorem}
	Let $f(x)$ and $g(x)$ be two functions for which the limits 
	$$ \lim_{x\to \infty}f(x)=F\quad \quad \quad \lim_{x\to \infty}=G$$
	exist. Then the following limits hold
	$$
	\lim_{x\to \infty} (f(x)+g(x))=F\pm G
	$$
	$$
	\lim_{x\to \infty}f(x)g(x)=FG
	$$
	$$
	\lim_{x\to \infty}\frac{f(x)}{g(x)}=\frac{F}{G} \quad \text{provided}~G\neq 0
	$$
	and for rational numbers $r$,
		$$\lim_{x\to \infty}(f(x))^r=F^r$$ provided that $f(x)^r$ is defined for all $x$.\\
		The analogous results hold for limits to $-\infty$.
\end{theorem}
\end{tcolorbox}
We need a little extra care  with the posers of functions.\\
\noindent {\bf Warning:} Consider that 
$$\lim_{x\to +\infty}\frac{1}{x^{1/2}}=0$$ However, 
$$\lim_{x\to +\infty}\frac{1}{(-x)^{1/2}}$$  does not exist because $x^{1/2}$ is not defined for $x<0$.

\begin{Example}
	Compute the following limit:
	$$\lim_{x\to \infty} \frac{x^2-3x+4}{3x^2+8x+1}$$
\end{Example}
\begin{solution}
	By factoring $x$ with largest exponent in the numerator and denominator we have
	$$\lim_{x\to \infty} \frac{x^2-3x+4}{3x^2+8x+1}=\lim_{x\to \infty} \frac{x^2( 1+\frac{-3x}{x^2}+\frac{4}{x^2} )}{x^2( 3+\frac{8x}{3x^2}+\frac{1}{3x^2} )}=\lim_{x\to \infty} \frac{( 1+\frac{-3x}{x^2}+\frac{4}{x^2} )}{( 3+\frac{8x}{3x^2}+\frac{1}{3x^2} )}=
	$$
	$$\frac{(\lim_{x\to \infty} 1+\lim_{x\to \infty}\frac{-3x}{x^2}+\lim_{x\to \infty}\frac{4}{x^2}) }{( \lim_{x\to \infty}3+\lim_{x\to \infty}\frac{8x}{3x^2}+\lim_{x\to \infty}\frac{1}{3x^2} )}=\frac{1}{3}.$$
\end{solution}
\begin{remark}
	Note that 
	$$\sqrt{x^2}=|x|=\begin{cases}
	x & x\geq 0\\
	-x & x<0.
	\end{cases}$$
	$$
	\includegraphics[width=1\linewidth]{lim25}
	$$
\end{remark}
\begin{Example}
	Compute the following limit:
	$$\lim_{x\to \infty} \frac{\sqrt{4x^2+1}}{5x-1}.$$
		$$
	\includegraphics[width=1\linewidth]{lim26}
	$$
\end{Example}

\begin{solution}
	Factor the terms with the largest exponents in the numerator and denominator. We have
	$$\lim_{x\to \infty} \frac{\sqrt{4x^2+1}}{5x-1}=\lim_{x\to \infty} \frac{\sqrt{4x^2(1+\frac{1}{4x^2})}}{5x(1-\frac{1}{5x})}=\lim_{x\to \infty} \frac{\sqrt{4x^2}\sqrt{(1+\frac{1}{4x^2})}}{5x(1-\frac{1}{5x})}=\lim_{x\to \infty} \frac{2|x|}{5x}=\lim_{x\to \infty} \frac{2x}{5x}=\frac{2}{5}.$$
\end{solution}



\begin{Example}
	Compute the following limit:
	$$\lim_{x\to -\infty} \frac{\sqrt{4x^2+1}}{5x-1}.$$
\end{Example}

\begin{solution}
	By the same kind of computation we have
		$$\lim_{x\to -\infty} \frac{\sqrt{4x^2+1}}{5x-1}=\lim_{x\to \infty} \frac{2|x|}{5x}.$$
	Consider that since $x$ is getting negative values, we have $|x|=-x$. Therefore,	
		$$\lim_{x\to -\infty} \frac{\sqrt{4x^2+1}}{5x-1}=\lim_{x\to \infty} \frac{2|x|}{5x}=\lim_{x\to \infty} \frac{-2x}{5x}=\frac{-2}{5}.$$
\end{solution}

\begin{Example}
	Compute the following limit:
	$$\lim_{x\to \infty} \left( x^{7/5}-x\right).$$
\end{Example}

\begin{solution}
	We factor the term with the largest exponent, we have
		$$\lim_{x\to \infty} \left( x^{7/5}-x\right)=\lim_{x\to \infty} x^{7/5}\left( 1-\frac{1}{x^{2/5}}\right)=\infty.$$
\end{solution}

\begin{theorem}
	Let $a,c,H\in \mathbb{R}$ and let $f,g,h$ be functions defined in an interval around $a$ (but they need not be defined at $x=a$), so that 
	$$ \lim_{x\to a}f(x)=+\infty \quad \quad \lim_{x\to a}g(x)=+\infty \quad \quad \lim_{x\to a}h(x)=H$$
	\begin{enumerate}
		\item $$\lim_{x\to a}(f(x)+g(x))=+\infty.$$
			\item $$\lim_{x\to a}(f(x)+h(x))=+\infty.$$
				\item $$\lim_{x\to a}(f(x)-g(x))=\text{undetermined}.$$
					\item $$\lim_{x\to a}(f(x)-h(x))=+\infty.$$
					\item $$\lim_{x\to a}cf(x)=\begin{cases}
					+\infty & c>0\\
					0 & c=0\\
					-\infty & c<0
					\end{cases}$$
					\item $$\lim(f(x).g(x))=+\infty.$$
					\item $$\lim_{x\to a} (f(x).h(x))
					=\begin{cases}
					+\infty & H>0\\
					\text{undetermined} & H=0\\
					-\infty & H<0
					\end{cases}$$
					\item $$\lim_{x\to a} \frac{h(x)}{f(x)}=0.$$
	\end{enumerate}
\end{theorem}

\begin{tcolorbox}[width=\textwidth,colback={green!20},title={},colbacktitle=yellow,coltitle=blue]
\begin{Example}
	Consider the following three functions:
	$$f(x)=x^{-2}\quad \quad g(x)=2x^{-2}\quad \quad h(x)=x^{-2}-1.$$
	Then 
	$$\lim_{x\to 0}f(x)=+\infty \quad \quad \lim_{x\to 0}g(x)=+\infty \quad \quad \lim_{x\to 0}h(x)=+\infty.$$
	Then 
   \begin{itemize}
   	\item $$ \lim_{x\to 0} (f(x)-g(x))=\lim_{x\to 0}x^{-2}=-\infty$$
   	\item $$ \lim_{x\to 0} (f(x)-h(x))=\lim_{x\to 0}(1)=1$$
   	\item $$ \lim_{x\to 0} (g(x)-h(x))=\lim_{x\to 0}x^{-2}+1=\infty$$
   \end{itemize}
\end{Example}
\end{tcolorbox}

\newpage

\section{Continuity}

Look at all the following functions.
$$
	\includegraphics[width=.3\linewidth]{x21}~	\includegraphics[width=.3\linewidth]{x3-2x+1}$$ $$	\includegraphics[width=.3\linewidth]{sinx}~\includegraphics[width=.3\linewidth]{cosx}~
$$
All of these functions are continuous. Roughly speaking, a function is continuous if it does not have any abrupt jumps. Now consider the following function.
	$$\includegraphics[width=.3\linewidth]{cont1}~~\includegraphics[width=.3\linewidth]{cont2} ~~~~\includegraphics[width=.3\linewidth]{cont3}$$
These functions are not continuous. The function $f$, $g$, and $h$ have abrupt jumps at $x=2$, $x=0$, and $x=1$, respectively, so $f$ is not continuous at $a$, $g$ is not continuous at $0$, and $h$ is not continuous at $1$.

\begin{tcolorbox}[width=\textwidth,colback={green!20},title={},colbacktitle=yellow,coltitle=blue] 
\begin{definition}
	A function $f(x)$ is continuous at $a$ if 
	$$\lim_{x\to a}f(x)=f(a).$$
If a function is not continuous at $a$ then it is said to be discontinuous at $a$. 
When we write that $f$ is continuous without specifying a point, then typically
this means that $f$ is continuous at $a$ for all $a\in \mathbb{R}$..
When we write that $f(x)$ is continuous on the open interval $(a, b)$ then the function
is continuous at every point $c$ satisfying $a< c<b$.
\end{definition}
\end{tcolorbox}
From the above definition we immediately have that if $f$ is continuous at $a$, then 
\begin{enumerate}
	\item $f(a)$ exists;
	\item $\lim_{x\to a^-}$ exists and is equal to $f(a)$.
		\item $\lim_{x\to a^+}$ exists and is equal to $f(a)$.
\end{enumerate}

\begin{tcolorbox}[width=\textwidth,colback={green!20},title={},colbacktitle=yellow,coltitle=blue] 
\begin{definition}
	A function is continuous from the left at $a$ if 
	$$\lim_{x\to a^-}=f(a).$$ And 
		a function is continuous from the right at $a$ if 
		$$\lim_{x\to a^-}=f(a).$$
\end{definition}
\end{tcolorbox}

\begin{tcolorbox}[width=\textwidth,colback={green!20},title={},colbacktitle=yellow,coltitle=blue] 
\begin{definition}
	A function $f(x)$ is continuous on an interval $[a,b]$  if 
	\begin{enumerate}
		\item $f(x)$ continuous on $(a,b)$,
		\item $f(x)$ is continuous form the right at $a$,
		\item $f(x)$ is continuous form the left at $b$.
	\end{enumerate}
\end{definition}

\begin{definition}
	A function $f(x)$ is continuous on an interval $(a,b]$  ({\color{red}{on the interval $[a,b)$}})  if 
	\begin{enumerate}
		\item $f(x)$ continuous on $(a,b)$,
		\item $f(x)$ is continuous form the left at $b$ ({\color{red}{from the right at $a$}}).
	\end{enumerate}
\end{definition}
\end{tcolorbox}

%\begin{definition}
%	A function $f(x)$ is continuous on an interval $[a,b)$   if 
%	\begin{enumerate}
%		\item $f(x)$ continuous on $(a,b)$,
%		\item $f(x)$ is continuous form the right at $a$.
%	\end{enumerate}
%\end{definition}
\begin{tcolorbox}[width=\textwidth,colback={green!20},title={},colbacktitle=yellow,coltitle=blue] 
\begin{Example}
	Consider the function 
	$$f(x)=\begin{cases}
	x & x<1\\
	x+2 & x\geq 1
	\end{cases}$$
	\begin{mdframed}
		$$\text{jump discontinuity}\quad \quad 
	\includegraphics[width=.2\linewidth]{cont1}
	$$
\end{mdframed}
	\begin{itemize}
		\item $$\lim_{x\to 1^{-}}f(x)=1~~~~~~\lim_{x\to 1^{+}}f(x)=3~~~~~~f(1)=3.$$
		\item The function $f(x)$, at $x=1$ is not continuous because the limit does not exist; however, it is continuous form the right at $1$ since 
		$$\lim_{x\to 1^{+}}f(x)=3=f(1).$$
		\item The function $f(x)$, on $[1,\infty)$ (for $x\geq 1$) is continuous.
		\item The function $f(x)$, on $(-\infty,-1)$ is continuous.
	\end{itemize}
\end{Example}
		\end{tcolorbox}


\begin{tcolorbox}[width=\textwidth,colback={green!20},title={},colbacktitle=yellow,coltitle=blue] 
\begin{Example}
				Consider the function 
	$$g(x)=\begin{cases}
	\frac{1}{x^2} & x\neq 0\\
	0 & x=0
	\end{cases}$$
\begin{mdframed}
	$$\text{infinite discontinuity}\quad \quad 
	\includegraphics[width=.4\linewidth]{cont2}
	$$
\end{mdframed}
\begin{itemize}
	\item Consider that 
	$$ \lim_{x\to 0^{-}}g(x)=\infty=\lim_{x\to 0^{+}}g(x)~~~~~~~g(0)=0.$$
Thus 	the function $g(x)$ is not continuous at $0$ 
because  $$\lim_{x\to 0}g(x)=\infty\neq 0=g(0).$$
It is not continuous at $0$ from the left since $\lim_{x\to 0^{-}}g(x)=\infty\neq 0=g(0)$ and not from  the right since $\lim_{x\to 0^{+}}g(x)=\infty\neq 0=g(0)$.
	\item the function $g(x)$ is continuous at all points in $\mathbb{R}$ except $0$. 
\end{itemize}

\end{Example}
\end{tcolorbox}

	\begin{tcolorbox}[width=\textwidth,colback={green!20},title={},colbacktitle=yellow,coltitle=blue] 
\begin{Example}
	Consider the function 
	$$h(x)=\begin{cases}
	\frac{x^3-x^2}{x-1} & x\neq 1\\
	0 & x=1
	\end{cases}$$
	\begin{mdframed}
		$$\text{removable discontinuity}\quad \quad 
		\includegraphics[width=.4\linewidth]{cont3}
		$$
	\end{mdframed}
	\begin{itemize}
		\item $$\lim_{x\to 1^{-}}h(x)=1=\lim_{x\to 1^{+}}h(x)~~~~~~~~~~~~f(1)=0.$$
		\item $$\lim_{x\to 1}=1.$$
		\item the function $h(x)$ is not continuous at $1$ since $$\lim_{x\to 1}h(x)=1\neq 0=h(1).$$ It is not continuous form the left since $$\lim_{x\to 1^{-}}h(x)=1\neq 0=h(1)$$ and not from the right since $$\lim_{x\to 1^{+}}h(x)=1\neq 0=h(1).$$
		\item the function $h(x)$ is continuous at all points in $\mathbb{R}$ except $1$. 
	\end{itemize}
\end{Example}
\end{tcolorbox}
%
%\begin{definition}
%	A function $f(x)$ is continuous from the right at $a$ if$$\lim_{x\to a^+}f(x)=f(a).$$
%	Similarly, a function $f(x)$ is continuous from the left at $a$ if 
%	$$\lim_{x\to a^-}=f(a).$$ A function is continuous at $a$ if it is continuous form the left and right at $a$, that is,
%	$$\lim_{x\to a^+}f(x)=f(a)=\lim_{x\to a^+}f(x).$$
%\end{definition}
\begin{tcolorbox}[width=\textwidth,colback={green!20},title={},colbacktitle=yellow,coltitle=blue] 
\begin{lemma}
	Let $c\in \mathbb{R}$. The functions 
	$$f(x)=x~~~~~~~~~~~~~g(x)=c$$
	are continuous everywhere on the real line.
\end{lemma}
\end{tcolorbox}

\begin{tcolorbox}[width=\textwidth,colback={green!20},title={},colbacktitle=yellow,coltitle=blue] 
\begin{theorem} ({\bf Arithmetic of continuity})
	Let $a,c\in \mathbb{R}$ and let $f(x)$ and $g(x)$ be functions that are continuous at $a$. Then the following functions are also continuous at $x=a$.
	\begin{itemize}
		\item $f(x)+g(x)$ and $f(x)-g(x)$,
		\item $cf(x)$ and $f(x)g(x)$, and 
		\item $\frac{f(x)}{g(x)}$ provided $g(a)\neq 0$. 
	\end{itemize}
\end{theorem}
\end{tcolorbox}


\begin{tcolorbox}[width=\textwidth,colback={green!20},title={},colbacktitle=yellow,coltitle=blue] 
	\begin{theorem} The following functions are continuous everywhere in their domains
		\begin{itemize}
			\item polynomials and rational functions (for example $f(x)=x^5+4x^2+1$ and $g(x)=\frac{x^2+1}{x+1}$)
			\item roots and powers (for example $h(x)=\sqrt{x}$ and $r(x)=2^x$)
						\item trig functions and their inverses (for example $k(x)=\sin(x)$ and $t(x)=\cos^{-1}(x)$)
			\item exponentials and logarithms (for example $s(x)=e^x$ and $q(x)=\ln x$). 
		\end{itemize}
	\end{theorem}
\end{tcolorbox}


\begin{tcolorbox}[width=\textwidth,colback={green!20},title={},colbacktitle=yellow,coltitle=blue] 
\begin{Example} Determine when the function $f(x)=\frac{\sin(x)}{x^2-5x+6}$ is continuous?
	Since both $\sin(x)$ and $x^2-5x+6$ are continuous by the above theorem we only need to check when $x^2-5x+6=0$. Note that $x^2-5x+6=(x-2)(x-3)$, thus this polynomial is only zero at $x=2$ and $x=3$. Therefore, $f(x)$ is continuous at all points in $\mathbb{R}$ except $2$ and $3$.
\end{Example}
\end{tcolorbox}


\begin{tcolorbox}[width=\textwidth,colback={green!20},title={},colbacktitle=yellow,coltitle=blue]
\begin{theorem}
	If $g$ is continuous at $a$ and $f(x)$ is continuous at $g(a)$, then $(f\circ g)(x)=f(g(x))$ is continuous at $x=a$. 
\end{theorem}
\end{tcolorbox}

\begin{tcolorbox}[width=\textwidth,colback={green!20},title={},colbacktitle=yellow,coltitle=blue]
\begin{Example}
	Determine when the function $h(x)=\sqrt{\sin(x)}$ is continuous. 
\end{Example}

\begin{solution}
	Let $f(x)=\sqrt{x}$ and $g(x)=\sin(x)$, then $h(x)=(f\circ g)(x)$. We only need to find out at what points $\sin(x)$ is positive.
	$$\includegraphics[width=1\linewidth]{sinexpanded}$$
	The function $\sqrt{\sin(x)}$ is continuous if 
	$$x\in [2n\pi,(2n+1)\pi]\quad \quad \text{for all natural numbers $n$}.$$
\end{solution}
\end{tcolorbox}

\begin{tcolorbox}[width=\textwidth,colback={green!20},title={},colbacktitle=yellow,coltitle=blue]
\begin{theorem}{\bf (Intermediate value theorem(IVT))} 
	Let $a<b$ and let $f(x)$ be a function that is continuous at all points $a\leq x\leq b$. If ${Y}$ is any number between $f(a)$ and $f(b)$ then there exists some number $c\in [a,b]$ so that $f(c)={Y}$.
\end{theorem}

$$
\includegraphics[width=.7\linewidth]{IVT}
$$
\end{tcolorbox}
\begin{remark}
One of the main application of the IVT theorem is showing a function $f$ has a zero inside an interval. For example, in the following picture
$$
\includegraphics[width=.7\linewidth]{IVT0}
$$
we can see that $f(a)<0$ and $f(b)>0$, therefore by IVT, there is a number $c$ between $a$ and $b$ such that $f(c)=0$.

\end{remark}
\begin{remark}
If $f$ is continuous and $f(a) \leq Y \leq f(b)$, the IVT merely shows that there is a $a \leq c \leq b$ such that $f(c)=Y$, but it doesn't show how many of them exist. For example, in the following picture, we can see $f(a) \leq Y \leq f(b)$, and there are three numbers $c_1,c_2$, and $c_3$ such that $f(c_1)=f(c_2)=f(c_3)=Y$.
$$
\includegraphics[width=.7\linewidth]{IVT4}
$$
\end{remark}

\begin{remark}
	Consider that if the function $f$ is not continuous at the interval $[a,b]$ then the IVT fails. In the following examples, even though $f(a)\leq Y\leq f(b)$, there is not a number $a \leq c\leq b$ such that $f(c)=Y$. 
	$$
	\includegraphics[width=1\linewidth]{IVT5}
	$$
\end{remark}
 
\begin{tcolorbox}[width=\textwidth,colback={green!20},title={},colbacktitle=yellow,coltitle=blue]
\begin{Example}
	Show that the function $f(x)=x-1+sin(\pi x /2)$ has a zero in $0\leq x \leq 1$.
\end{Example}

\begin{solution}
	Consider that $f(x)$ is a continuous function such  that $f(0)=-1$ and $f(1)=1$. Therefore, by IVT, since $f(0)=-1\leq 0 \leq 1 = f(1)$, we have $f(c)=0$ for some $c\in [0,1]$.
\end{solution}
\end{tcolorbox}

\begin{Example}
	Use the bisection method to find a zero of 
	$f(x)=x-1+\sin(\pi x/2)$ that lies between $0$ and $1$.
\end{Example}

\begin{solution}
	\begin{itemize}
		\item 
	Let $a=0$ and $b=1$. Then 
	$$f(0)=-1$$
	$$f(1)=1$$
	\item Test the point in the middle $x=\frac{1-0}{2}=0.5$, 
	$$f(0.5)=0.2071067813>0$$
	\item Let $a=0$ and $b=0.5$. Then 
	$$ f(0)=-1 $$
	$$ f(1)= 0.2071067813$$
	So by IVT, there is a zero in $[0,0.5]$.
	\item Test the point in the middle $x=\frac{0.5-0}{2}=0.25.$
	$$ f(0.25)=-0.3673165675 <0.$$
	\item Let $a = 0.25$, $b = 0.5$ where $f (0.25)<0$ and $f (0.5)>0$. By IVT there is a zero in the interval $[0.25,0.5]$.
	\item So without much work we know the location of a zero inside a range of length $1/4$. Each iteration will halve the length of the range and we keep going until we reach the precision we need, though it is much easier to program a computer to do it.
\end{itemize}
\end{solution}
%
%\begin{solution}
%	For learning about bisection method and how to solve this problem refer to Example 1.6.15 of the CLP book. 
%\end{solution}

\chapter{Derivatives}

\section{Revisiting Tangent Lines}

\begin{tcolorbox}[width=\textwidth,colback={green!20},title={},colbacktitle=yellow,coltitle=blue]
\begin{Example}
	Find the slope of the tangent line to the curve $y=x^2$ that passes through $P=(1,1)$.
\end{Example}
$$
\includegraphics[width=1\linewidth]{yx2}
$$

\begin{solution}
	Consider that the slope of the secant line is 
	$$ \frac{f(1+h)-f(1)}{(1+h)-1}=\frac{f(1+h)-f(1)}{h}.$$
	And the slope of the tangent line is the same as 
	$$\lim_{h\to 0}\frac{f(1+h)-f(1)}{h}.$$
\end{solution}
\end{tcolorbox}

\begin{tcolorbox}[width=\textwidth,colback={green!20},title={},colbacktitle=yellow,coltitle=blue]
\begin{theorem}
Given a function $f(x)$ the slope of the tangent line at $x=a$ (if exists) is 
$$\lim_{x\to a} \frac{f(a+h)-f(a)}{h}.$$
\end{theorem}
\end{tcolorbox}

%$$
%\begin{tikzpicture}
%\begin{axis}[
%    axis lines = left,
%    xlabel = $x$,
%    ylabel = {$f(x)$},
%]
%\addplot [
%    domain=0:2, 
%    samples=5, 
%    color=red, thick
%]
%{x^2 };
%\addlegendentry{$x^2 $}
%
%\addplot [
%    domain=0:2, 
%    samples=4, 
%    color=blue, thick
%]
%{x+1};
%
%\end{axis}
%\end{tikzpicture}
%$$

\section{Definition of the derivative}

\begin{tcolorbox}[width=\textwidth,colback={green!20},title={},colbacktitle=yellow,coltitle=blue]
\begin{definition}(Derivative at a point)
Let $a\in\mathbb{R}$ and let $f(x)$ be a function defined on an open interval that contains $a$.
\begin{itemize}
\item The derivative of $f(x)$ at $x=a$ is denoted $f'(a)$ and is defined by 
\begin{equation}\label{derivative} f'(a)=\lim_{h\to 0}\frac{f(a+h)-f(a)}{h} \end{equation} if the limit exists. 
\item  When the above limit exists, the function $f(x)$ is said to be differentiable at $x=a$. When the limit 
does not exist, the function $f(x)$ is said to be not differentiable at $x=a$. 
\item We can equivalently define the derivative $f'(a)$  by the limit 
$$f'(a)=\lim_{x\to a} \frac{f(x)-f(a)}{x-a}.$$ 
\end{itemize}
To see that these two definitions are the same, we set $x=a+h (x-a=h)$ and then when $h$ approaches $0$, we have 
$x$ approaches $a$, and the limit in \ref{derivative} becomes 
$\lim_{x\to a} \frac{f(x)-f(a)}{x-a}.$
\end{definition}
\end{tcolorbox}
\begin{tcolorbox}[width=\textwidth,colback={green!20},title={},colbacktitle=yellow,coltitle=blue]
\begin{Example}
Let $a,c\in \mathbb{R}$ be constants. Compute the derivative of the function $f(x)=c$ at $x=a$. 
\end{Example}

\begin{solution}
By the definition of the derivative, we have
\begin{equation*}
\begin{split}
f'(a)=&\lim_{h\to 0}\frac{f(a+h)-f(a)}{h}\\
&= \lim_{h\to 0}\frac{c-c}{h}\\
&= \lim_{h\to 0}0\\
&=0
\end{split}
\end{equation*}
\end{solution}
\end{tcolorbox}
\begin{tcolorbox}[width=\textwidth,colback={green!20},title={},colbacktitle=yellow,coltitle=blue]
\begin{Example}
Let $a\in \mathbb{R}$. Compute the limit of the function $g(x)=x$ at $x=a$.
\end{Example}

\begin{solution}
By the definition of the derivative we have 
\begin{equation*}
\begin{split}
f'(a)=&\lim_{h\to 0}\frac{f(a+h)-f(a)}{h}\\
& =\lim_{h\to 0}\frac{(a+h)-a}{h}\\
& =\lim_{h\to 0}\frac{h}{h}\\
&=\lim_{h\to 0} 1\\
&=1. 
\end{split}
\end{equation*}
\end{solution}
\end{tcolorbox}

We have so proved our first theorem which is the following.
\begin{tcolorbox}[width=\textwidth,colback={green!20},title={},colbacktitle=yellow,coltitle=blue]
\begin{theorem}(easiest derivative)
Let $a,c\in \mathbb{R}$ and let $f(x)=c$ and $g(x)=x$. Then 
$$f'(a)=0$$
and 
$$g'(a)=1.$$
\end{theorem}
\end{tcolorbox}
\begin{tcolorbox}[width=\textwidth,colback={green!20},title={},colbacktitle=yellow,coltitle=blue]
\begin{Example}
Compute the derivative of $f(t)=t^2$ at $t=a$.
\end{Example}

\begin{solution}
We have that 
\begin{equation*}
\begin{split}
f'(a)&=\lim_{h\to 0}\frac{f(a+h)-f(a)}{h}\\
&=\lim_{h\to 0}\frac{(a+h)^2-a^2}{h}\\
&=\lim_{h\to 0}\frac{a^2+h^2+2ah-a^2}{h}\\
&=\lim_{h\to 0}\frac{h^2+2ah}{h}\\
&=\lim_{h\to 0}\frac{h(h+2a)}{h}\\
&=\lim_{h\to 0}h+2a\\
&=2a
\end{split}
\end{equation*}	
\end{solution}
\end{tcolorbox}














\newpage

\begin{tcolorbox}[width=\textwidth,colback={red!30},title={},colbacktitle=yellow,coltitle=blue] 
\noindent {\bf Homework:}
\end{tcolorbox}

Go to this link\\ \href{https://www.mooculus.osu.edu/textbook/mooculus.pdf}{https://www.mooculus.osu.edu/textbook/mooculus.pdf} and download the book "MOOCULUS". Then do the following questions:
\begin{itemize}
	\item all questions in page 35;
	\item in page 33 see why $\lim_{x\to 0}\frac{sin(x)}{x}=1$. Then do Questions 1-8 page 38;
	\item in page 42, do questions 1-10.
\end{itemize}



\begin{thebibliography}{9}
	\bibitem{CLP}  CLP1: Differential Calculus by J. Feldman, A. Rechnitzer, and E. Yeager.
	
\end{thebibliography}





%\newpage
%
%\begin{theorem}[Franco]
%	Let ${\rm BP}(M)$ be the matroid basis polytope for the polytope $M$. Then two distinct vertices $e_A$ and $e_B$ form an edge of ${\rm BP}(M)$ if and only if 
%	 $C,B\in M$ with $e_A+e_B=e_C+e_D$ implies that $\{A,B\}=\{C,D\}$.
%\end{theorem}
%
%\begin{theorem}
%	Let $M$ be a matroid. Two distinct vertices $e_A$ and $e_B$ form an edge of
%	the matroid independence polytope
%of $M$ if and only if 
%	$C,B\in M$ with $e_A+e_B=e_C+e_D$ implies that $\{A,B\}=\{C,D\}$.
%\end{theorem}
%
%\begin{proof}
%	It is clear that if $e_A$ and $e_B$ are form an edge, then  $C,B\in M$ with $e_A+e_B=e_C+e_D$ implies that $\{A,B\}=\{C,D\}$.
%	
%	Now we want to show that if 
%	$C,B\in M$ with $e_A+e_B=e_C+e_D$ implies that $\{A,B\}=\{C,D\}$, then $e_A$ and $e_B$ form an edge. We proceed the proof by contradiction. Suppose that if 
%	$C,B\in M$ with $e_A+e_B=e_C+e_D$ implies that $\{A,B\}=\{C,D\}$ and $\{e_A,e_B\}$ is not an edge. Therefore, there are $C_1,\ldots,C_k$ distinct from $A$ and $B$ such that 
%	\begin{equation}\label{adjacent-condition}
%	e_A-e_B=\sum_{i=1}^k \gamma_i(e_{C_i}-e_B).
%	\end{equation}
%	We establish the proof by analyzing the cardinalities of $A$ and $B$. 
%	
%	If $|A|>|B|+1$, then there exists $a\in A\setminus B$ such that $B\cup \{a\}$ is in $M$; moreover, $A\setminus \{a\}$ is in $M$. So by letting $C=B\cup \{a\}$ and $D=A\setminus \{a\}$, we have $$e_A+e_B=e_C+e_D, $$ a contradiction.
%	
%	Therefore, we have two cases $|A|=|B|+1$ or $|A|=|B|$. 
%	
%	{\bf Case 1:} $|A|=|B|+1.$ Suppose that $|A\setminus B|>1$. Then there exists $a\in A\setminus B$ such that $B\cup \{a\}$ is in $M$; moreover, $A\setminus \{a\}$ is in $M$. Let $C=B\cup \{a\}$ and $D=A\setminus \{a\}$. We have $$e_A+e_B=e_C+e_D, $$ a contradiction. Therefore, we may assume that $|A\setminus B|=1$, and consequently, $B\subseteq A$. By investigating  (\ref{adjacent-condition}), we conclude that each $C_i$ is equal to $A$ or $B$, a contradiction. 
%	
%	{\bf Case 2:}  $|A|=|B|=r$. Consider (\ref{adjacent-condition}). Note that each $C_i$ is a subset of $A\cup B$.  Assume that there exists $C_i$ such that $|C_i|>r$. Then there exists $x\in C_i\setminus A$ such that $A\cup \{x\}\in M$. Since $C_i\subseteq A\cup B$, it follows that $x\in B$. Let $C=A\cup \{x\}$ and $D=B\setminus \{x\}$. Then
%	$$e_A+e_B=e_C+e_D, $$ a contradiction.  Consequently, all $C_i$'s have the same  cardinality $r$.  Remark that all elements of rank $r$ of $M$ produces a matroid basis polytope, say $P$. Therefore, all $C_i's$ are vertices of $P$.  In the matroid basis polytope $P$, simultaneously 
%	\begin{itemize}
%		\item  if $e_C$ and $e_D$ are vertices of $P$ and $e_A+e_B=e_C+e_D$ implies that $\{A,B\}=\{C,D\}$; and
%		\item there are $C_1,\ldots,C_k$ distinct from $A$ and $B$ such that  $e_A-e_B=\sum_{i=1}^k \gamma_i(e_{C_i}-e_B),$
%	\end{itemize}
%which is a contradiction because the former, by the previous theorem , means $\{e_A,e_B\}$ forms an edge in the matroid basis polytope $P$, and the latter means there is not edge between $e_A$ and $e_B$ in $P$.
%\end{proof}


\end{document}